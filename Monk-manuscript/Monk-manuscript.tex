\documentclass[]{elsarticle} %review=doublespace preprint=single 5p=2 column
%%% Begin My package additions %%%%%%%%%%%%%%%%%%%
\usepackage[hyphens]{url}

  \journal{Fisheries Research} % Sets Journal name


\usepackage{lineno} % add
\providecommand{\tightlist}{%
  \setlength{\itemsep}{0pt}\setlength{\parskip}{0pt}}

\usepackage{graphicx}
%%%%%%%%%%%%%%%% end my additions to header

\usepackage[T1]{fontenc}
\usepackage{lmodern}
\usepackage{amssymb,amsmath}
\usepackage{ifxetex,ifluatex}
\usepackage{fixltx2e} % provides \textsubscript
% use upquote if available, for straight quotes in verbatim environments
\IfFileExists{upquote.sty}{\usepackage{upquote}}{}
\ifnum 0\ifxetex 1\fi\ifluatex 1\fi=0 % if pdftex
  \usepackage[utf8]{inputenc}
\else % if luatex or xelatex
  \usepackage{fontspec}
  \ifxetex
    \usepackage{xltxtra,xunicode}
  \fi
  \defaultfontfeatures{Mapping=tex-text,Scale=MatchLowercase}
  \newcommand{\euro}{€}
\fi
% use microtype if available
\IfFileExists{microtype.sty}{\usepackage{microtype}}{}
\bibliographystyle{elsarticle-harv}
\ifxetex
  \usepackage[setpagesize=false, % page size defined by xetex
              unicode=false, % unicode breaks when used with xetex
              xetex]{hyperref}
\else
  \usepackage[unicode=true]{hyperref}
\fi
\hypersetup{breaklinks=true,
            bookmarks=true,
            pdfauthor={},
            pdftitle={Comparison of indices of relative abundance from recreational fishery-dependent surveys},
            colorlinks=false,
            urlcolor=blue,
            linkcolor=magenta,
            pdfborder={0 0 0}}
\urlstyle{same}  % don't use monospace font for urls

\setcounter{secnumdepth}{0}
% Pandoc toggle for numbering sections (defaults to be off)
\setcounter{secnumdepth}{0}

% Pandoc citation processing
\newlength{\cslhangindent}
\setlength{\cslhangindent}{1.5em}
\newlength{\csllabelwidth}
\setlength{\csllabelwidth}{3em}
% for Pandoc 2.8 to 2.10.1
\newenvironment{cslreferences}%
  {}%
  {\par}
% For Pandoc 2.11+
\newenvironment{CSLReferences}[2] % #1 hanging-ident, #2 entry spacing
 {% don't indent paragraphs
  \setlength{\parindent}{0pt}
  % turn on hanging indent if param 1 is 1
  \ifodd #1 \everypar{\setlength{\hangindent}{\cslhangindent}}\ignorespaces\fi
  % set entry spacing
  \ifnum #2 > 0
  \setlength{\parskip}{#2\baselineskip}
  \fi
 }%
 {}
\usepackage{calc}
\newcommand{\CSLBlock}[1]{#1\hfill\break}
\newcommand{\CSLLeftMargin}[1]{\parbox[t]{\csllabelwidth}{#1}}
\newcommand{\CSLRightInline}[1]{\parbox[t]{\linewidth - \csllabelwidth}{#1}\break}
\newcommand{\CSLIndent}[1]{\hspace{\cslhangindent}#1}

% Pandoc header



\begin{document}
\begin{frontmatter}

  \title{Comparison of indices of relative abundance from recreational
fishery-dependent surveys}
    \author[Southwest Fisheries Science Center]{Melissa Hedges Monk}
   \ead{melissa.monk@noaa.gov} 
    \author[University of California Santa Cruz]{Rebecca R. Miller}
   \ead{rebecca.miller@noaa.gov} 
    \author[Another University]{E.J. Dick}
   \ead{edward.dick@noaa.gov} 
    \author[Southwest Fisheries Science Center]{Derek Zoolander}
   \ead{derek@example.com} 
      \address[Southwest Fisheries Science Center]{Fisheries Ecology
Division, Southwest Fisheries Science Center, National Marine Fisheries
Service, NOAA, 110 McAllister Way, Santa Cruz, CA 95066}
    \address[University of California Santa Cruz]{Department, Street,
City, State, Zip}
    
  \begin{abstract}
  This is the abstract.

  It consists of two paragraphs.
  \end{abstract}
  
 \end{frontmatter}

\hypertarget{introduction}{%
\section{Introduction}\label{introduction}}

Fisheries stock assessments rely on a wide range of data to model a
population to fully understand the dynamics. When fishery-independent
data are not available, assessors try to make best used of the best
available data, which may often only include fishery-dependent data. The
West Coast groundfish Fishery Management Plan (FMP) includes 64 rockfish
species, \_\_\_ of which do not have full stock assessments. Fisheries
survey and catch data are used to develop standardized indices of
abundance that inform fisheries stock assessment models (Maunder and
Punt, 2004). Catch per unit effort (CPUE) is a common metric collected
from fishery-dependent or fishery-independent surveys, with the latter
providing unbiased data. However, fishery-independent surveys can be
costly, labor intensive and often require a long time series to be
considered informative in fisheries stock assessments. Advantages of
fishery-dependent data is that it is collected directly from the the
fishery whose operations are not constrained by sampling designs, but
dependent on the behaviors of the captain and, in the case of
recreational trips, customer preference. Fishery-dependent are only
collected from areas legally open areas can be collected, i.e., areas
closed to fishing are not sampled. In California, this includes a
network of marine protected areas (MPAs), rockfish conservation areas
(RCAs) developed based on depth closures, and varying seasonal and depth
closures that vary temporally and spatially along California's
coastline. Fishery-independent surveys are conducted using a scientific
study design and, depending on the study, are not always confined to the
same regulations as the commercial and recreational fishing sectors. In
an ideal situation, both fishery-dependent and fishery-independent
surveys would used to inform the stock assessment model.

Depending on the stock assessment model and the available data for the
fish stock, the index can have large influence on the assessment results
(find examples).

There are only two long-term fishery-independent surveys that span the
U.S. West Coast that survey groundfish populations and neither are
informative for nearshore rockfish species. The National Marine
Fisheries Service Southwest Fisheries Science center conducts a juvenile
rockfish and ecosystem survey using a midwater trawl and the West Coast
Bottomfish Trawl (WCGBT) survey utilizes a xxx trawl to sample
groundfish populations deeper than 55 m. However, many of the important
rockfish species inhabit untrawlable habitat and in depths shallower
than the 55 m sampled by the WCGBT survey. The California Collaborative
Fisheries Research Program (CCFRP), a fishery-independent hook-and-line
survey, was designed to monitor California's network of Marine Protected
Areas. The CCFRP was limited to the central coast of California from
2007-2016, and expanded to cover the entire California coast in 2017.
Indices of abundance developed from this program have informed nearshore
rockfish stock assessments in recent years (\textbf{Monk2019?};
\textbf{Monk2021?}). The CCFRP is not without its limitations in terms
of spatial coverage and

An index of relative abundance assumes that changes in the index are
proportional to changes of abundance in the population, which does not
always hold or may not be linear (\textbf{Harley2001?}).

Along the U.S. West Coast, even if the stock assessment is categorized
as data rich, oftentimes the the only index of abundance available is
from a fishery-dependent CPUE time series of observed recreational
angler catch rates (Cope, 2013).

There is currently no coastwide fishery-independent survey for the
nearshore fisheries on the West Coast. The National Marine Fisheries
Service (NMFS) conducts an annual trawl survey, but the survey extends
to a minimum depth of \_\_\_ (cite) and is known not to select for all
rockfish species due to the inability to trawl in rocky habitats.
Several scientists are developing methods to survey untrawlable habitat
and small areas of fishery-independent survey work are available.
Propose survey methods include the use of stereo video landers (cite),
acoustic surveys (cite), remotely operated vehicles (ROVs), as well as
hook and line surveys. There are three ongoing hook and line surveys on
the West Coast, two of which were designed to monitor MPA networks in
central California (CCFRP) and Oregon (ODFW Marine Reserves Program). T

In California, approximately 70\% of the recreational fishing effort is
in the SCB, and since the survey effort distributed proportional to
effort, there are a large number of fishery-dependent data available for
the SCB that overlap with the NWFSC hook and line survey fixed stations.

This data series is part of the onboard observer program, which collects
location- and species-specific CPUE information for the recreational
fishing fleet (Monk et al.~2014)

Fishery-dependent surveys sample the fishing fleets and are subject to
potential sampling biases. The sampling is dependent on the fishing
boat's behavior, which is to maximize catch. Sampling of the fishing
fleet is often opportunistic based on the availability of samplers and
the availability of trips to sample. Sampling the fisheries can also be
constrained to the current regulations, which may prohibit the retention
of a species or fishing at certain depths, i.e., California Department
of Fish and Wildlife has varying spatial and temporal depth and season
closures implemented through six management regions. There is also a
fairly new network of Marine Protected Areas (MPAs) designated from
2007-2012 that prohibit recreational fishing, and are therefore areas no
longer sampled by the recreational fishing fleet. However, the advantage
to fishery-dependent sampling the reduced program cost compared to a
more intensive scientific fishery-independent survey.

Fishing effort directed toward a specific species or species complex

This paper focuses on methods to determine the appropriate trips to
represent effort and the resulting indices of relative abundance from a
fishery-dependent survey of the recreational fishing fleet. We apply
multiple methods to learn how abundance indices change with available
information, as well as look at the inclusion of supplement data
collected by Cal Poly. Cal Poly began an onboard observer program
following the same methods as CDFW in 2001.

\hypertarget{groundfish-management}{%
\paragraph{Groundfish Management}\label{groundfish-management}}

There are a number of management changes that have affected West Coast
groundfish species over the last few decades. In 2000, West Coast
groundfish fisheries were declared a disaster (citation) as a number of
stocks were declared overfished with overfishing occurring. In response
to federal management measures, the CDFW implemented fishing gear
regulations of 3-hooks per line and one-line maximum per angler in 2000,
and that changed to a 2-hook per line and one-line maximum in 2001. CDFW
also bagan temporal and spatial management in the forms of closures and
depth restrictions in 2004. Boundaries of the regional management units
have changed over time and are accounted for in our analyses. In 2007,
California enacted a network of Marine Protected Areas that encompass
approximately 23\% of state waters. As groundfish stock have rebuilt
over the last few years, CDFW eased depth restrictions in some
management areas and the recreational groundfish fleet had access to
depths outsdie of state waters.

\hypertarget{methods}{%
\section{Methods}\label{methods}}

\hypertarget{survey-data}{%
\paragraph{Survey Data}\label{survey-data}}

The California Department of Fish and Wildlife (CDFW) has conducted a
fishery-dependent onboard observer survey of the Commercial Passenger
Fishing Vessel (CPFV or party/charter boat) fleet since 1999. Since
2004, the survey became part of the California Recreational Fisheries
Survey (CFRS). Groundfish-targeted CPFV trips were sampled
opportunistically as CPFV participation is voluntary and sampling effort
was distributed in proportion to fishing effort. In California, xx\% of
the recreational CPFV effort is north of Point Conception. Observers may
not be allowed on a vessel if the vessel is at full capacity, which is
more common in northern California where a number of charter boats are
smaller 6-pack vessels with limited capacity.

On a trip, observers recorded information for each fishing drop, each
time lines were in the water. Just prior to the start of each fishing
drop, the sampler selected a subset of anglers to observe, at maximum of
15 anglers per fishing drop. The sampler recorded all fish encountered
(retained and discarded) by the subset of anglers as a group. Samplers
also recorded the time fished (starting when the captain announced
``Lines down'' to when the captain instructed anglers to reel lines up),
GPS coordinates of the fishing drop (start and/or end
latitude/longitude), and minimum and maximum bottom depth. Fish
encountered by the group of observed anglers were recorded to the
species level as either retained or discarded, providing a count of each
species at a particular location. Discarded fish were measured for
length and some portion of retained fish were measured as part of a
different CRFS Sampling program. The catch and fishing time of an
individual angler were not recorded. Additional details can be found in
Monk et al. (\textbf{Monk2014?}).

In 2001, the California Polytechnic State University Institute of Marine
Science, San Luis Obispo (Cal Poly) conducts a similar onboard observer
program of the CPFV fleet based in Port Avila and Port San Luis along
the Central Coast. Protocols for the Cal Poly survey are the same as the
CDFW survey, with the exception that Cal Poly measures retained and
discarded fish from observed anglers.

A common phenomenon of ecological data is the high proportion of zero
observations across samples and the question as to whether the sampling
occurred within the species' habitat and the species was not observed or
if the sampling occurred outside of the species' habitat (structural
zeroes). Fisheries survey data are often subset to exclude structural
zeroes using the Stephens-MacCall method, which looks at the species
composition of co-occurring species. However, the onboard observer
survey collected location-specific information on each observer fish
encounter. To subset the onboard observer survey data and exclude
structural zeroes, we used the positive catch locations as a proxy for
suitable habitat.

\hypertarget{species}{%
\paragraph{Species}\label{species}}

We explored the methods described in the following sections indices of
abundance for fourteen species or species complexes of management
interest: black rockfish (\emph{Sebastes melanops}), blue and deacon
rockfish complex (\emph{Sebastes mystinus}, \emph{Sebastes diaconus}),
brown rockfish (\emph{Sebastes auriculatus}), China rockfish
(\emph{Sebastes nebulosus}), copper rockfish (\emph{Sebastes caurinus}),
gopher and black-and-yellow rockfish complex (\emph{Sebastes carnatus},
\emph{Sebastes chrysomelas}), greenspotted rockfish (\emph{Sebastes
chlorostictus}), olive rockfish (\emph{Sebastes serranoides}), quillback
rockfish (\emph{Sebastes maliger}), rosy rockfish (\emph{Sebastes
rosaceus}), starry rockfish (\emph{Sebastes constellatus}), vermilion
and sunset rockfish complex (\emph{Sebastes miniatus}/\emph{Sebastes
crocotulus}), yellowtail rockfish (\emph{Sebastes flavidus}). Species
complexes consist of two cryptic species that may or may not be
genetically distinct, but cannot be assessed separately for various
reasons including the inability to separate catch histories between
species or difficulty of visual species identification. Versions of the
area-weighted habitat index of relative abundance were approved by the
Pacific Fisheries Management Council's SSC for use in stock assessments
in 2013 have been used in xxx assessments accepted for management
(China, gopher/black-and-yellow, vermilion/sunset, blue/deacon, black,
lingcod - cite assessments).

\hypertarget{treatment-of-data}{%
\paragraph{Treatment of Data}\label{treatment-of-data}}

The onboard observer data provide a high-resolution of catch, effort and
the ability to map the fishing drops to fine-scale habitat data. This
paper explores methodological differences in data treatment to see what
we gain by having the high-resolution data. To do this, we first mimic
the collection of dockside data by aggregating all of the fish
encountered within a single trip and summing the effort among drifts.
Trip level data were then filtered using the Stephens-MacCall approach
and three different data selection methods were applied using the
Stephens-MacCall results (see description below), and only county was
used as a spatial covariate in the indices.

The second approach used the high resolution trip data, but assumed no
available habitat data. The percent of groundfish encountered during a
drift was assumed as a proxy for habitat.

The third approached used the fishing drop level data, incorporated
habitat as a filter for data selection, and applied an area-weighted
index using the area of the reef within a region as a proxy for habitat.

In addition, all of these approaches were applied with and with out the
supplemental data from the Cal Poly observer program to illustrate the
effect of additional data on indices for species with a distirbution
centered in central California.

All indices of abundance were coded in R and the Bayesian analyses were
conducted using the rstanarm package.

Analyses were limited to the California coast north of Point Conception
(\(34^\circ 27^\prime N\)). The composition of the fish communities in
southern California differ, and the recreational fisheries are
fundamentally different, with a higher percentage of trips targeting
mixed species and pelagic and highly migratory species, as well as more
limited access to rocky habitat nearshore. Point Conception is a
biogeographic break (citation) and a number of stock assessments In
addition, complete habitat data are not available for areas in southern
California. The data were also temporally restricted to the years
2001-2016. Earlier and more recent data were excluded to preserve a
dataset with the most consistent gera and depth regulations.

\hypertarget{stephens-maccall-filering}{%
\subparagraph{Stephens-MacCall
Filering}\label{stephens-maccall-filering}}

The Stephens-MacCall (2004) filtering approach was used to predict the
probability of encountering a target species, based on the species
composition of the catch in a given trip. The method uses
presence/absence data wthin a logistic regression to identify the
proability of encountering a target species given the presence or abance
of other predictor species. This method is commonly used to filter data
that are collected dockside after a vessel returns to port. Prior to
applying the Stephens-MacCall filter, we identified potentially
informative predictor species, i.e., species with sufficient sample
sizes and temporal coverage (present in at least 5\% of all trips) to
inform the binomial model. The remaining species all co-occurred with
the target species in at least one trip and were retained for the
Stephens-MacCall logistic regression. Coefficients from the
Stephens-MacCall analysis (a binomial GLM) are positive for species that
are more likely to co-occur with the target species, and negative for
species that are less likely to be caught with target species.

While the filter is useful in identifying co-occurring or non-occurring
species assuming all effort was exerted in pursuit of a single target,
the targeting of more than one species or species complex (``mixed
trips'') can result in co-occurrence of species in the catch that do not
truly co-occur in terms of habitat associations informative for an index
of abundance. Stephens and MacCall (2004) recommended including all
trips above a threshold where the false negatives and false positives
are equally balanced. However, this does not have any biological
relevance and for this data set, and we assume that if the target
species was encountered, the vessel fished in appropriate habitat.

Stephens and MacCall (2004) proposed filtering (excluding) trips from
the index standardization based on a criterion of balancing the number
of false positives and false negatives. False positives (FP) are trips
that are predicted to encounter the target species based on the species
composition of the catch, but did not. False negatives (FN) are trips
that were not predicted to encounter the target species, given the catch
composition, but caught at least one. The trips selected using this
criteria were compared to an alternative method including all the
``false positive'' trips, regardless of the probability of encountering
the target species. The catch included in this index and in the dockside
data collected by CDFW is sampler-examined and the samplers are well
trained in species identification. Therefore, we make the assumption
that species were positively and correctly identified. Three data
selection methods were applied to the Stephens-MacCall method , the data
selection method proposed in the original manuscript to balance the
false negatives and the false positives, retention of all positive
encounters and exclude of only false negatives, and the method described
in (xxxx).

\hypertarget{indices-of-abundance}{%
\paragraph{Indices of Abundance}\label{indices-of-abundance}}

Standardized indices of abundance were generated for each data filtering
method. Indices of abundance modeled the catch per unit effort (angler
hours) and possible covariates trip-level data were 3-month wave and
county of landing. Covariates considerd for the drop-level data
included, aggregated reef area, 3-month wave, depth, and xxxx.

All indices were modeled using a Bayesian genearlized linear models
(GLMs). Species with high positive encounter rates were modeled with a
negative biniomals

The onboard observer data were analyzed using the delta method with two
generalized linear models (delta-GLM). The first GLM models the
probability of encountering the species of interest with a binomial
likelihood and a logit link function. The second GLM models the positive
encounters with either gamma or lognormal errors structure.

We explored the possibility of area-weighted indices, using the area of
the reefs as the weighting scheme.

\hypertarget{habitat-data}{%
\paragraph{Habitat Data}\label{habitat-data}}

We identified rocky habitat and defined reefs as potential habitat for
rockfish in California from multiple bathymetic data sources. Bathymetry
within California state waters north of Point Conception
(\(34^\circ 27^\prime N\)) was mapped at a resolution of 2 m by the
California Seafloor Mapping Program (CSMP).Rough and smooth substrate
was identified by CSMP using 2 rugosity indices based upon bathymetric
data, surface:planar area, and vector ruggedness measure (VRM). We
considered areas identified as `rough' as reef habitat. While there were
fishing drops outside of state waters, we limited data for the
comparisons presented in this paper to state waters with known habitat.

Individual reefs at the finest scale were defined as raster cells of
rough habitat greater than 200 m apart. The distance was chosen based on
evidence that a number of nearshore rockfish exchit site fidelit and a
number of tagging studies have recaptured close to original capture
sites (Hannah et al., 2012; Hannah and Rankin, 2011; Lea et al., 1999;
\textbf{Matthew1990?}). If raster cells representing hard substrate were
contiguous (not separated by soft habitat by greater than 200 m) it
remained intact, no matter how large the reef. Reefs were further
defined with a 5 m buffer to account for potential error in positional
accuracy.

Fishing drops were assigned to reefs based on the recorded start
location, given that the end locations were not always available. Reefs
within predetermined larger regions were designated to gain appropriate
sample sizes needed for modelling and the areas of the hard habitat were
summed.

\hypertarget{results}{%
\section{Results}\label{results}}

\hypertarget{discussion}{%
\section{Discussion}\label{discussion}}

Recent studies have identified the need to investigate the assumptions
and uncertainty in relative indices of abundance from visual surveys
(Bacheler and Shertzer 2015, Campbell et al.~2015, Schobernd et
al.~2013) and simulation studies (Siegfried et al.~2016).

Linear relationship between density from SCUBA surveys and a
fishery-independent survey for rockfish when in the species preferred
habitat and gear (Haggarty and King 2006) and for the more abundance
species (Richards and Schnute 1986).

Magnusson and Hilborn 2007 - prioritize data for stock assessments

Stock synthesis weighting of indices based on CVs - is the CV tighter
for the fishery-independent survey to give it have an edge over the
onboard observer survey?

Starr et al. (2010) compared SCUBA methods to commercial fishing gear,
but only sampled for two year.

CDFW sampler manual - ``10 anglers should be the target number of
observed anglers''

encompass the entire range of the species. However, the point of the
exercise is to compare the two methods and these surveys are sampling
the same habitats in the SCB

Survey indices can be either absolute or relative. In the case of an
absolute index of abundance, the entire population within the sampling
area is accounted for and the index also provides information on the
density of the fish species within that area as well as aid in scaling
the population size within the stock assessment model. Most indices of
abundance are relative due to the fact that the entire population within
the survey area was not observed. Estimates of absolute abundance are
difficult to obtain, especially for cryptic rockfishes. The cowcod
(\emph{Sebastes levis}) stock assessments is one of the only West Coast
stock assessments that has incorporated an estimate of absolute
abundance, derived from a visual survey {[}(\textbf{Piner?}) et
al.~2005). The majority of stock assessments include one or more index
of relative abundance.

Composition data from recreational surveys had the largest impact on
simulation results, but individual survey components did not have
individual effects on benchmarks (Siegfried et al.~2016).\\
The onboard observer surveys decrease the amount of uncertainty, but
relative to a fishery-independent survey, is still high\ldots.

Assume fishing behavior is remains the same when observers are no
onboard the vessel.

\hypertarget{references}{%
\section*{References}\label{references}}
\addcontentsline{toc}{section}{References}

\hypertarget{refs}{}
\begin{CSLReferences}{1}{0}
\leavevmode\vadjust pre{\hypertarget{ref-Cope2013}{}}%
Cope, J.M., 2013. {Implementing a statistical catch-at-age model (stock
synthesis) as a tool for deriving overfishing limits in data-limited
situations}. Fisheries Research 142, 3--14.
doi:\href{https://doi.org/10.1016/j.fishres.2012.03.006}{10.1016/j.fishres.2012.03.006}

\leavevmode\vadjust pre{\hypertarget{ref-Hannah2011}{}}%
Hannah, R.W., Rankin, P.S., 2011. {Site fidelity and movement of eight
species of pacific rockfish at a high-relief rocky reef on the oregon
coast}. North American Journal of Fisheries Management 31, 483--494.
doi:\href{https://doi.org/10.1080/02755947.2011.591239}{10.1080/02755947.2011.591239}

\leavevmode\vadjust pre{\hypertarget{ref-Hannah2012}{}}%
Hannah, R.W., Rankin, P.S., Blume, M.T.O., 2012. {Use of a novel cage
system to measure postrecompression survival of Northeast Pacific
rockfish}. Marine and Coastal Fisheries 4, 46--56.
doi:\href{https://doi.org/10.1080/19425120.2012.655849}{10.1080/19425120.2012.655849}

\leavevmode\vadjust pre{\hypertarget{ref-Lea1999}{}}%
Lea, R.N., McAllister, R.D., VenTresca, D.A., 1999. {Biological aspects
of nearshore rockfishes of the genus
genus\(\backslash\)emph{\{}Sebastes{\}} from central California: with
notes on ecologically related sport fishes}. Fish Bulletin No. 177 112.

\leavevmode\vadjust pre{\hypertarget{ref-Maunder2004}{}}%
Maunder, M.N., Punt, A.E., 2004. {Standardizing catch and effort data: A
review of recent approaches}. Fisheries Research 70, 141--159.
doi:\href{https://doi.org/10.1016/j.fishres.2004.08.002}{10.1016/j.fishres.2004.08.002}

\leavevmode\vadjust pre{\hypertarget{ref-Starr2010}{}}%
Starr, R.M., Carr, M., Malone, D., Greenley, A., McMillan, S., 2010.
{Complementary sampling methods to inform ecosystem-based management of
nearshore fisheries}. Marine and Coastal Fisheries: Dynamics,
Management, and Ecosystem Science 2, 159--179.
doi:\href{https://doi.org/10.1577/C08-056.1}{10.1577/C08-056.1}

\leavevmode\vadjust pre{\hypertarget{ref-Stephens2004}{}}%
Stephens, A., MacCall, A., 2004. {A multispecies approach to subsetting
logbook data for purposes of estimating CPUE}. Fisheries Research 70,
299--310.
doi:\href{https://doi.org/10.1016/j.fishres.2004.08.009}{10.1016/j.fishres.2004.08.009}

\end{CSLReferences}


\end{document}
