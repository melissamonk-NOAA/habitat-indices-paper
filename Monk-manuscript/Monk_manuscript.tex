% Options for packages loaded elsewhere
\PassOptionsToPackage{unicode}{hyperref}
\PassOptionsToPackage{hyphens}{url}
\PassOptionsToPackage{dvipsnames,svgnames,x11names}{xcolor}
%
\documentclass[
  12pt,
  authoryear,
  preprint,
  3p]{elsarticle}

\usepackage{amsmath,amssymb}
\usepackage{iftex}
\ifPDFTeX
  \usepackage[T1]{fontenc}
  \usepackage[utf8]{inputenc}
  \usepackage{textcomp} % provide euro and other symbols
\else % if luatex or xetex
  \usepackage{unicode-math}
  \defaultfontfeatures{Scale=MatchLowercase}
  \defaultfontfeatures[\rmfamily]{Ligatures=TeX,Scale=1}
\fi
\usepackage{lmodern}
\ifPDFTeX\else  
    % xetex/luatex font selection
\fi
% Use upquote if available, for straight quotes in verbatim environments
\IfFileExists{upquote.sty}{\usepackage{upquote}}{}
\IfFileExists{microtype.sty}{% use microtype if available
  \usepackage[]{microtype}
  \UseMicrotypeSet[protrusion]{basicmath} % disable protrusion for tt fonts
}{}
\makeatletter
\@ifundefined{KOMAClassName}{% if non-KOMA class
  \IfFileExists{parskip.sty}{%
    \usepackage{parskip}
  }{% else
    \setlength{\parindent}{0pt}
    \setlength{\parskip}{6pt plus 2pt minus 1pt}}
}{% if KOMA class
  \KOMAoptions{parskip=half}}
\makeatother
\usepackage{xcolor}
\setlength{\emergencystretch}{3em} % prevent overfull lines
\setcounter{secnumdepth}{5}
% Make \paragraph and \subparagraph free-standing
\ifx\paragraph\undefined\else
  \let\oldparagraph\paragraph
  \renewcommand{\paragraph}[1]{\oldparagraph{#1}\mbox{}}
\fi
\ifx\subparagraph\undefined\else
  \let\oldsubparagraph\subparagraph
  \renewcommand{\subparagraph}[1]{\oldsubparagraph{#1}\mbox{}}
\fi


\providecommand{\tightlist}{%
  \setlength{\itemsep}{0pt}\setlength{\parskip}{0pt}}\usepackage{longtable,booktabs,array}
\usepackage{calc} % for calculating minipage widths
% Correct order of tables after \paragraph or \subparagraph
\usepackage{etoolbox}
\makeatletter
\patchcmd\longtable{\par}{\if@noskipsec\mbox{}\fi\par}{}{}
\makeatother
% Allow footnotes in longtable head/foot
\IfFileExists{footnotehyper.sty}{\usepackage{footnotehyper}}{\usepackage{footnote}}
\makesavenoteenv{longtable}
\usepackage{graphicx}
\makeatletter
\def\maxwidth{\ifdim\Gin@nat@width>\linewidth\linewidth\else\Gin@nat@width\fi}
\def\maxheight{\ifdim\Gin@nat@height>\textheight\textheight\else\Gin@nat@height\fi}
\makeatother
% Scale images if necessary, so that they will not overflow the page
% margins by default, and it is still possible to overwrite the defaults
% using explicit options in \includegraphics[width, height, ...]{}
\setkeys{Gin}{width=\maxwidth,height=\maxheight,keepaspectratio}
% Set default figure placement to htbp
\makeatletter
\def\fps@figure{htbp}
\makeatother

\usepackage{booktabs}
\usepackage{longtable}
\usepackage{array}
\usepackage{multirow}
\usepackage{wrapfig}
\usepackage{float}
\usepackage{colortbl}
\usepackage{pdflscape}
\usepackage{tabu}
\usepackage{threeparttable}
\usepackage{threeparttablex}
\usepackage[normalem]{ulem}
\usepackage{makecell}
\usepackage{xcolor}
\usepackage{placeins}
\usepackage{setspace}
\usepackage{lineno}
\onehalfspacing
\linespread{2}
\linenumbers
\makeatletter
\makeatother
\makeatletter
\makeatother
\makeatletter
\@ifpackageloaded{caption}{}{\usepackage{caption}}
\AtBeginDocument{%
\ifdefined\contentsname
  \renewcommand*\contentsname{Table of contents}
\else
  \newcommand\contentsname{Table of contents}
\fi
\ifdefined\listfigurename
  \renewcommand*\listfigurename{List of Figures}
\else
  \newcommand\listfigurename{List of Figures}
\fi
\ifdefined\listtablename
  \renewcommand*\listtablename{List of Tables}
\else
  \newcommand\listtablename{List of Tables}
\fi
\ifdefined\figurename
  \renewcommand*\figurename{Figure}
\else
  \newcommand\figurename{Figure}
\fi
\ifdefined\tablename
  \renewcommand*\tablename{Table}
\else
  \newcommand\tablename{Table}
\fi
}
\@ifpackageloaded{float}{}{\usepackage{float}}
\floatstyle{ruled}
\@ifundefined{c@chapter}{\newfloat{codelisting}{h}{lop}}{\newfloat{codelisting}{h}{lop}[chapter]}
\floatname{codelisting}{Listing}
\newcommand*\listoflistings{\listof{codelisting}{List of Listings}}
\makeatother
\makeatletter
\@ifpackageloaded{caption}{}{\usepackage{caption}}
\@ifpackageloaded{subcaption}{}{\usepackage{subcaption}}
\makeatother
\makeatletter
\@ifpackageloaded{tcolorbox}{}{\usepackage[skins,breakable]{tcolorbox}}
\makeatother
\makeatletter
\@ifundefined{shadecolor}{\definecolor{shadecolor}{rgb}{.97, .97, .97}}
\makeatother
\makeatletter
\makeatother
\makeatletter
\makeatother
\journal{Fisheries Research}
\ifLuaTeX
  \usepackage{selnolig}  % disable illegal ligatures
\fi
\usepackage[]{natbib}
\bibliographystyle{elsarticle-harv}
\IfFileExists{bookmark.sty}{\usepackage{bookmark}}{\usepackage{hyperref}}
\IfFileExists{xurl.sty}{\usepackage{xurl}}{} % add URL line breaks if available
\urlstyle{same} % disable monospaced font for URLs
\hypersetup{
  pdftitle={Methods to incorporate known habitat in indices of abundance for rocky reef associated species and applications to management},
  pdfauthor={Melissa Hedges Monk; Rebecca R. Miller; E.J. Dick; Grant Waltz; Dean Wendt},
  pdfkeywords={fisheries dependent data, habitat
association, groundfish, index of abundance},
  colorlinks=true,
  linkcolor={blue},
  filecolor={Maroon},
  citecolor={Blue},
  urlcolor={Blue},
  pdfcreator={LaTeX via pandoc}}

\setlength{\parindent}{6pt}
\begin{document}

\begin{frontmatter}
\title{Methods to incorporate known habitat in indices of abundance for
rocky reef associated species and applications to management}
\author[1]{Melissa Hedges Monk%
\corref{cor1}%
\fnref{fn1}}
 \ead{melissa.monk@noaa.gov} 
\author[2]{Rebecca R. Miller%
%
}
 \ead{rebecca.miller@noaa.gov} 
\author[1]{E.J. Dick%
%
}
 \ead{edward.dick@noaa.gov} 
\author[33]{Grant Waltz%
%
}
 \ead{cat@example.com} 
\author[3]{Dean Wendt%
%
}
 \ead{cat@example.com} 

\affiliation[1]{organization={Southwest Fisheries Science
Center}, addressline={110 McAllister Way}, city={Santa
Cruz}, country={}, postcode={95060}}

\affiliation[2]{organization={University of California Santa
Cruz}, addressline={Street Address}, city={Santa
Cruz}, country={}, postcode={95060}}

\affiliation[3]{organization={California Polytechnic State
University}, addressline={Street Address}, city={San Luis
Obispo}, country={}, postcode={93407}}


\cortext[cor1]{Corresponding author}
\fntext[fn1]{This is the first author footnote.}




        
\begin{abstract}
Indices of abundance developed from fishery-dependent data are typically
subject to a number of assumptions about the area and habitat fished due
to the aggreation of the catch at the level of a fishing trip.

In California, two surveys occur onboard the recreational charter boats
fleet and samplers record location-specific data on the catch and effort
during individual fishing f stops throughout a trip. This location
specific information coupled with high-resolution maps of the bottom
substrate allowed us to subset the survey data to areas of rocky reef
habitat. The six species of rockfish (\emph{Sebastes} spp.) modeled in
this paper as example all have high affinity to rocky habitat. We
compared the indices of abundance developed with and without including
information on rocky reef habitat. The identification of the rocky reefs
also allowed us to weight the index of abundance by the area of
available habitat with predefined regions. We show that in general the
finer scale trends are variable and weighting the indices by the amount
of habitat within finer scale area decrease the variance around the
estimates. We also show how the estimates of available habitat can be
used to portion catches across management areas.
\end{abstract}





\begin{keyword}
    fisheries dependent data \sep habitat
association \sep groundfish \sep 
    index of abundance
\end{keyword}
\end{frontmatter}\ifdefined\Shaded\renewenvironment{Shaded}{\begin{tcolorbox}[breakable, sharp corners, interior hidden, boxrule=0pt, borderline west={3pt}{0pt}{shadecolor}, frame hidden, enhanced]}{\end{tcolorbox}}\fi

\hypertarget{introduction}{%
\section{Introduction}\label{introduction}}

Information on the known area of available habitat provides unique
opportunities for incorporation into the data processing of stock
assessments and management decisions. We present methods and examples
for incorporating the area of available habitat for reef-associated
species within indices of abundance for stock assessments and allocation
of yield for management decisions.

It is not often the case where high-resolution habitat data and fishing
location information are both available, and for many fishery-dependent
surveys an analyst will have to determine which subset of the data to
use based on available information. The availability of a 20+ year time
series of onboard observer data from California CPFVs, coupled with
high-resolution habitat maps of rocky reef habitat, provides us with an
opportunity to evaluate the effectiveness of the Stephens-MacCall method
and compare standardized indices of abundance derived from data sets
that differ in terms of spatial and temporal resolution.

One of the major recreational targets in California are groundfish
species, a group of which includes dozens of rockfish species
(\emph{Sebastes spp.}). Rockfish species' affinity to rocky habitat
differs by species and ranges from the territorial gopher rockfish
(\emph{S. carnatus}) that resides within rocky crevices and maintains
small home ranges of less than 20\(m^2\) \citep{Larson:1980:TBB}, to a
the schooling, mid-water black rockfish (\emph{S. melanops}) that haa an
average home range of 0.25\(km^2\), and exhibits diurnal movement
offshore \citep{Green:2011:MSA}. Tagged black rockfish have also been
recovered hundreds of miles from the initial capture location
(California Collaborative Fisheries Research Program, unpublished data).
The association of \emph{Sebastes} with rocky habitat makes them ideal
candidates for exploring the ability to predict effective effort from
fishery-dependent data based on known habitat. In addition, the area of
known rocky reef habitat data creates an opportunity to weight the index
of abundance by the calculated area of rocky reef habitat along the
California coast.

Indices of abundance are commonly used to provide a stock assessment
model with information about the stock's trend over time
\citep{Harley:2001:CUE, Hilborn:1992:QFS}. For many fish stocks, only
fishery-dependent survey data are available. Fishery-dependent survey
data are more readily available than fishery-independent scientific
survey data due to factors including the lower cost to collect data,
more frequent sampling opportunities, and ability to collect data at
large spatial scales where the fisheries operate.

Modelling fishery-dependent data requires making a number of assumptions
due to the nature of the data being reliant on the behavior of the
fishing fleet. A common metric for modelling fishery-dependent data is
catch per unit effort (CPUE), which is often used under the assumption
that the estimated trends are proportional to the true abundance of the
stock \citep{Maunder:2004:SCE}. However, catch rates are more likely to
reflect local densities than total abundance
\citep{Haggarty:2006:CIR, Schnute:1995:IEP}, in which case standardized
trends in CPUE (relative density) should be multiplied by habitat area,
when available, to estimate trends in relative abundance. Additionally,
fishery-dependent data are reliant on the behavior of the fishermen and
must be standardized to account for spatial and temporal changes in
fishing activity \citep{Campbell:2004:CSA, Hilborn:1992:QFS}.

An analyst must also consider factors such as the targeting of multiple
species when developing an index of abundance. The recreational for-hire
partyboat fleet may target multiple species during a trip. The target
species for a recreational trip is dependent on a number of factors
including weather that could limit transit to some fishing grounds, bag
limit regulations, angler preference and experience, duration of the
trip, and the captain's experience level. All of these factors affect
the catch during a trip. For example, a recreational trip in California,
USA may set out to target a particular rockfish (\emph{Sebastes spp.})
species associated with hard substrate, but if fishing is unsuccessful
or if bag limits are reached, the captain may spend a portion of the
trip targeting sanddab species (\emph{Citharichthys spp}) that inhabit
areas of soft substrate.

Therefore, an analyst must determine which samples within a survey
represent the effective effort directed towards the target species. The
granularity of the calculation of fishing effort is dependent on the
survey. A survey that interviews an angler or group of anglers at the
dock or pier after the fishing trip concludes provides fishing effort at
the level angler-days or angler hours. On the opposite end of the
spectrum is an onboard observer survey (onboard survey) where a sampler
rides along during a trip and records information on the catch and
effort, often from a subset of anglers, at every fishing location during
the trip. For these data, both temporal resolution (trip vs.~drift) and
spatial resolution (location of landing vs.~location of fishing) are
improved.

Here we focus on data available from fishery-dependent onboard observer
surveys of California's recreational partyboat fleet, also referred to
as commercial passenger fishing vessels (CPFV). The onboard observer
data provide an opportunity to explore what information we gain from
explicit knowledge of fishing locations. There are two surveys of the
California recreational CPFV fleet \citep{Monk:2014:DRD}. The California
Department of Fish and Wildlife (CDFW) surveys active ports throughout
the state and the California Polytechnic State University San Luis
Obispo (Cal Poly) surveys vessels with home ports in San Luis Obispo
County. In addition, we are able to utilize high resolution bathymetric
data to define appropriate habitat for a target species.

We utilized the onboard survey data to create develop methods that
account for the available habitat within fine scale areas. We utilized
the fishing drift level data with known location from the onboard
observer surveys and subset data based on the proximity to rocky reef
habitat. For this data set we also evaluated the effect of weighting by
reef habitat area. We applied these methods across six nearshore
rockfish species with different life histories, habitat preferences and
commonness in the data.

Stock assessment models estimate an Overfishing Limit (OFL) for a single
stock or a sub-area of a stock. On the West Coast of the U.S., the
Pacific Fishery Management Councils currently manages the nearshore
rockfish complex based on their depth distributions and at a
biogeographic break near Cape Mendocino, California ( 40 10), which
means the yield produced in California is divided into two separate
management areas.

found in the waters off California are typically modelled recognizing
the biogeographic and CDFW management breaks at Point Conception and
Cape Mendocino (reference map). Fisheries data are collected
independently by each state (Washignton, Oregon, and California) on the
fisheries in the waters off their respective coasts. Stock assessment
boundaries may be drawn at political boundaries and not at the
management boundaries. This creates a question on how to divide the OFL
between areas.

We applied weights to sub-areas of habitat across California for and
also calculated the amount of available habitat within and outside
California's network of Marine Protected Areas to account for the area
within the area open and closed to fishing from a fishery-independent
survey. The amount of availale habitat by management areas that do not
align with the spatial extent of a stock structure allowed us to
decompose the Annual Catch Limit (ACL) by management area.

We utilized two data sets, one fishery-independent an one
fishery-dependent that represent two methods of incorporating known
habitat data in the development of indices of abundance. The
fihsery-dependent data source is a combination of a CDFW and Cal Poly
surveys of the recreational CPFV fleet that fishes with hook-and-line
and targets groundfish species. The fishery-independent data source is
an MPA monitoring survey along the central coast of California, the
California Collaborative Fisheries Reserach Program (CCFRP). The CCFRP
expanded to cover additional monitoring areas in 2017, but we limit the
the data in this paper to the core sampling area. The incorportaion of
habitat data is made possible by the California Seafloor Mapping Program
(CSMP) that collected bathymetry and interpretted the data for nearly
all of California's state waters.

A number of methods are currently used to allocate an OFL with the stock
boundary does not align with a management boundary. Allocation of the
OFL could, ideally, be based on a fishery-independent survey of
abundance, but lacking that information several alternatives exist.
Previous allocations have used catch as a proxy for abundance when no
other information was available (Dick and MacCall, 2010; Dick et
al.~2011). Allocation of the OFL proportional to catch works if the
catch is proportional to biomass, which is unlikely in many cases.
Allocation based on catch may also allocate catch to areas where harvest
excedes the OFL. When fishery-independent survey data are avaialble
allocation can be based on estimates of survey biomass. This requires
that the survey covers the entire area of the stock assessment, which is
not often the case.

Copper rockfish (\emph{Sebastes caurinus}) is a nearshore species
ranging from northern Baja Mexico to the Gulf of Alaska. They inhabit
sub-tidal depths as junveile as juveniles and are commonly encountered
in depths up to 180 m as adults\citep{Love:2002:RNP, Love:1996:XXX}.
Copper rockfish prefer low-relief rocky habitat and have high site
fidelity. These characteristics make copper rockfish an ideal species to
illustrate the benefits of incorporating known habitat. Additional
examples and more details on each of the datasets we utlize are
available in a number of groundfish stock assessments (cite
assessments).

\textbf{keep or remove? We also examined the assumption that effort
might be proportional to reef area by obtaining effort estimates for the
CPFV trips (cite RecFIN) and comparing the area of rocky reef habitat to
the effort in each region.}

\hypertarget{methods}{%
\section{Methods}\label{methods}}

We present methods to define areas of rocky habitat within California's
state waters. We then describe how the known habitat allows us to filter
fishery-dependent data with known fishing locations and comparisons of
indices of abundance with and without accounting the area of known
habitat. Using a fishery-independent survey that monitors California's
network of marine protected areas we demonstrate the ability to account
for closed areas in an index of abundance. Lastly, we present and
compare four methods for allocating yield with and without consideration
of available habitat.

\hypertarget{mapping-and-identification-of-rocky-reefs}{%
\subsection{Mapping and Identification of Rocky
Reefs}\label{mapping-and-identification-of-rocky-reefs}}

We predicted rocky reef habitat using high resolution seafloor mapping
of California state waters (out ot 3nm) north of Point Conception
California (34.45). The California Seafloor Mapping Project (CSMP)
collected bathymetry and backscatter data collected from 20xx to 20xx
\citep{Golden:2013:CSW, CSUMB:2014:CSM}. The CSMP mapped the mainland
California state waters at a 2 m resolution from the California-Mexico
border to the California-Oregon border. The mapped area does not include
very shallow areas close to shore (the ``white zone''), which extend
approximately 200-500 m from the shoreline. Maps of the white zone are
not yet available and we assume proportionality in the area of habitat
in the white zone and the mapped area.

We created a mosaic from 137 CSMP substrate blocks that ranged in size
from 16 \(\mathrm{km}^2\) to more than 400 \(\mathrm{km}^2\). The CSMP
identified rough and smooth substrates, surface:planar area, and a
vector ruggedness measure (VRM) of the bathymetric digital elevation
model {[}\#fig-map2{]}. The CSMP set a varying VRM threshold for each of
the substrate blocks, removed any artifacts, and the product is
considered a conservative estimate of rough habitat.

We converted the digital mosaic of 137 CSMP substrate raster blocks with
pixels designated as rough habitat (our rocky reef habitat proxy) from a
raster format to polygons. We applied a 5 m buffer region around the
rough habitat polygon to allow for any small errors in positional
accuracy. Contiguous polygons of rocky reef substrate were treated as a
single rocky reef, regardless of size. Polygons separated by 200 m were
treated as separate reefs; the 200 m cutoff is based upon expert
knowledge and a consensus that it represented a distance that most
rockfish species would traverse over sandy habitat. We calculated the
area of each reef defined and retained those greater than or equal to
\(100 \mathrm{m^2}\). We conducted all spatial analyses and overlay of
the data sources on the rocky reef habitat in ArcMap 10.3 (ESRI
citation).

The data in this paper included only areas north of Point Conception
(\(34^\circ 27^\prime N\)) due to gaps in the bathymetric layers to
interpret habitat coverage further south. Point Conception is a
significant biogeographic boundary (\citet{Valentine:1966:NAM}), and the
composition of the fish communities in southern California differ,
potentially reducing the effectiveness of methods that rely on species
associations, such as the method of Stephens and MacCall
(\citet{Stephens:2004:MAS}). The recreational fisheries south of Point
Conception are also fundamentally different, with a higher percentage of
CPFV trips targeting mixed species and pelagic and highly migratory
species. The availability of accessible rocky habitat in southern
California is also less in the southern California nearshore areas
compared to northern California.

\hypertarget{fishery-dependent-onboard-observer-survey}{%
\subsection{Fishery-Dependent Onboard Observer
Survey}\label{fishery-dependent-onboard-observer-survey}}

The CDFW began a fishery-dependent onboard observer survey of the
recreational CPFV fleet in 1999. In 2004, the CDFW integrated it into
their California Recreational Fisheries Survey (CFRS), designed to
estimate catch and effort. In response to a request from the fishing
industry, Cal Poly San Luis Obispo began a supplemental onboard observer
survey in 2001 of the CPFV fleet based in Port Avila and Port San Luis
along California's Central Coast {[}\#fig-map{]}. Both the CDFW and the
Cal Poly onboard observer surveys continue through present day.
Management of California's recreational fisheries is complex and has
undergone spatial and temporal changes since 1999. Between 1999 and
2003, the recreational regulations evolved from no restriction on the
number of lines or hooks an angler could deploy to a one line and
two-hook maximum, as well as implementation of depth restrictions that
vary across the coast. Subsequent management and recovery of overfished
species allowed a relaxation of depth restrictions beginning in 2017
that shifted fishing effort relative to the 2004-2016 period
\citep{Monk:2021:SVR}.

The National Marine Fishery Service's Southwest Fisheries Science Center
(SWFSC) developed a relational database for the CDFW onboard observer
survey \citeyearpar{Monk:2014:DRD} for the years 1999-2011 and is
updated annually with data from the CDFW. The Cal Poly data are also
provided to the SWFSC annually. All data were checked for potential
errors at the drift-level by SWFSC staff. The data sets were filtered
for errors within the relational database before analyses were
conducted, and the data used here reflect changes from the QA/QC process
that may not be reflected in the raw data available directly from the
CDFW.

While only a small portion of the total CPFV trips taken are sampled as
part of the onboard observer survey, the onboard observer survey
collects a large amount of data during each trip. During each trip the
sampler records information for each fishing drift, defined as a period
starting when the captain announces ``lines down'' to when the captain
instructs anglers to reel their lines up. Just prior to the start of
each fishing drift, the sampler selects a subset of anglers to observe,
at a maximum of 15 anglers per drift. The sampler records all fish
encountered (retained and discarded) by the subset of anglers as a
group, i.e., catch cannot be attributed to an individual angler.
Samplers also record the start and end times of a drift, location of the
fishing drift (start latitude/longitude and for most drifts, end
latitude/longitude), and minimum and maximum bottom depth. Fish
encountered by the group of observed anglers are recorded as either
retained or discarded. This provides information on the catch (count of
each species) and effort (time and number of anglers fished) during each
fishing drift. While both surveys include records of discarded fish, we
only used the retained catch in these analyses. Discarded fish can often
represent a different size structure than retained fish, either due to
size limits or angler preference, or represent fish encountered during a
temporal or spatial closure.

We assigned survey locations to rocky reefs based on the recorded start
location of a drift, given that the end locations were not always
recorded. We calculated the cumulative distribution of distance to rocky
reef (in meters) for drifts that retained copper rockfish with a
distance cutoff of 90\%. We applied further data filters to address
possible errors in the data. We removed drifts in the upper and lower
1\% of the recorded time fished and recorded observed anglers, as these
may not accurately define a successful fishing drift or may represent
data entry errors. Similarly, we filtered the data to retain 99\% of all
drifts based on average drift depth. We calculated average depth from
the recorded minimum and maximum depths when available or the imputed
minimum and maximum depth from the bathymetry layer described below.

\hypertarget{fishery-independent-mpa-monitoring-survey}{%
\subsection{Fishery-Independent MPA Monitoring
Survey}\label{fishery-independent-mpa-monitoring-survey}}

The California Collaborative Fisheries Research Program (CCFRP) is a
fishery-independent hook-and-line survey designed to monitor nearshore
fish populations at a series of sampling locations both inside and
adjacent to MPAs
\citep{starr_variation_2015a, wendt_collaborative_2009}. The CCFRP
survey began in 2007 along the central coast of California and was
designed in collaboration with academics, NMFS scientists and fishermen.
From 2007-2016 the CCFRP project was focused on the central California
coast, and has monitored four MPAs consistently. In 2017, the CCFRP
expanded coastwide within California. The index of abundance was
developed from the four MPAs sampled consistently (Año Nuevo and Point
Lobos by Moss Landing Marine Labs; Point Buchon and Piedras Blancas by
Cal Poly).

The survey design for CCFRP consists 500 x 500 m cells both within and
adjacent to each MPA. On any given survey day site cells are randomly
selected within a stratum (MPA and/or reference cells). CPFVs are
chartered for the survey and the fishing captain is allowed to search
within the cell for a fishing location. During a sampling event, each
cell is fished for a total of 30-45 minutes by volunteer anglers. Each
fish encountered is recorded, measured, and released (or descended to
depth) and can later be linked back to a particular angler. The CCFRP
samples shallower depths to avoid barotrauma-induced mortality. Starting
in 2017, a subset of fish have been retained to collect otoliths and fin
clips that provide needed biological information for nearshore species.
For the index of abundance, CPUE was modeled at the level of the drift,
similar to the fishery-dependent onboard observer survey described
above. The CCFRP data are quality controlled at the time data were
entered by each project partner.

\hypertarget{indices-of-abundance}{%
\subsection{Indices of Abundance}\label{indices-of-abundance}}

We generated standardized indices of abundance for copper rockfish from
the fishery-dependent onboard observer data and the fishery-independent
CCFRP data. For each, we developed an index wihout any habitat weighting
and one that weighted the the index by the amount of available habitat.
We modeled all indices using Bayesian generalized linear models (GLMs)
and the delta GLM method \citep{Lo:1992:IRA, Stefansson:1996:AGS}. The
delta GLM method is commonly used to standardize catch per unit effort
for stock assessments {[}citations{]}. The delta method models the data
with two separate GLMs; one for the probability of encountering the
species of interest using a binomial likelihood and a logit link
function, and the second GLM for the positive encounters assuming either
a gamma or lognormal error structure. The error structure of the
positive model was selected via the Akaike Information Criterion (AIC)
from models with the full suite of considered explanatory variables.

The response variable for the positive models was angler-retained catch
per unit effort. The full suite explanatory variables modeled and model
selection processes can be found in the supplemental material. Year was
always included in as an explanatory variable in model selection, even
if it was not significant, because the goal of the index of abundance
was to extract the year effect. The area-weighted index included a
year/rocky reef interaction term, even if it was not statistically
significant, to allow us to weight the index by the area of rocky reef.

Model selection for the binomial and positive observation models was
based on AIC using the lme4 package in R, and unless very different
predictors were selected, the same predictors were used in each of the
two Bayesian models. The Bayesian models were run with 5,000 iterations
and weakly informative priors. Posterior predictive model checks were
examined for both the binomial and positive observation models,
including the predicted percent positive compared to the maximum
likelihood estimates. We constructed the unweighted annual index by
multiplying the back-transformed posterior draws from the year
coefficients from the binomial model by the exponentiated positive model
draws from the year coefficients, and taking the mean and standard
deviation of the distribution of the product for each year.

The area-weighted, habitat-informed index was developed by extracting
the posterior draws of the unweighted index, and then summing across the
product of the back-transformed posteriors weighted by the fraction of
total area within each reef. To compare the indices across the three
data filtering methods and the area-weighted index, each index was
scaled to its mean value.

\hypertarget{allocating-yield-to-management-areas}{%
\subsection{Allocating Yield to Management
Areas}\label{allocating-yield-to-management-areas}}

We present four methods to allocate yield for copper rockfish at the
management boundary near Cape Mendocino, California (40-10). These
include mehtods referred to as catch-based, habitat-based, CPUE-only and
habitat-weighted CPUE.

The catch-based method is derived using from the total estimated catch
for the time period 2020-2022 (can change) north of Point Conception in
California. The proportion north and south of Cape Mendocino equals the
allocation of yield. For the habitat-based method, if we assume the
average density of copper rockfish is constant over the assessed area
(Point Conception to the California/Oregon border), the fraction of
copper rockfish occurring north of Cape Mendocino would be equal to the
fraction of habitat in the same area.

The assumption of equal density may not be accurate, and no direct
estimates of density are available from a fishery-independent survey
with adequate spatial coverage. We propose an alternative method that
combines existing habitat information with a proxy for fish density,
using catch per unit effort. Although data from the onboard CPFV
observer surveys are more precise in terms of total catch, effort, and
location, the sampling rate north of Cape Mendocino does not provide
adequate data given the number of 6-pack vessels in northern California.
Sampling coverage for the CRFS dockside survey is spatially more
complete, in that numerous samples exist in the northern management
area. Therefore used the private boat CPUE data to develop an index of
abundance for the CPUE-only method, assuming CPUE is proportional to
density. For the habitat-weighted CPUE method, we multiplied the
area-specific CPUE estimates by the area of habitat to produce a spatial
index of relative abundance. Data were filtered using the same methods
detailed in the 2023 copper rockfish stock assessment for the CRFS
private boat dockside index. We compare the allocations of yield among
these four methods.

\textbf{still from blue - but a placehold - Years prior to 2013 were
subsequently dropped, to create an index that is representative of
recent catch rates in each area. Sample sizes (number of trips) for the
final data set are shown in Table D3. }

\hypertarget{results}{%
\section{Results}\label{results}}

\hypertarget{survey-data-and-rocky-reef-habitat}{%
\subsection{Survey Data and Rocky Reef
Habitat}\label{survey-data-and-rocky-reef-habitat}}

Prior to any filtering a total of 19,425 drifts that aggregated to 2,270
trips were available for the analyses. Approximately 21\% of all the
CPFV trips observed by CDFW from 2004-2016 occurred north of Point
Conception. It is important to note that north of Bodega Bay,
California, the majority of charter boats are smaller six-pack vessels
that may not have the capacity to carry a sampler onboard. As a result,
sample sizes in this part of the state are smaller than areas to the
south. The addition of the Cal Poly onboard observer survey to the CDFW
survey more than doubled the sample sizes of observed trips in San Luis
Obispo County, with an average annual increase of 155\% from 2004-2016.

From 2004-2016 the drift-level data contained a total of 19,425 fishing
drifts, and after removing drifts with missing effort information (time
fished and/or observed anglers), 19,180 drifts remained. The filter
removing the upper and lower 1\% of the time fished and number of
observed anglers resulted in fishing drifts lasting between three and 96
minutes and three to 15 observed anglers, and reduced the data to 18,591
fishing drifts. The remaining data filter for depth resulted in a cutoff
of 46.6 fathoms, and retained 18,405 drifts based on average drift
depth. A filter on the minimum depth was not included here because the
recreational fleet was not limited to a minimum fishing depth and all of
the fishing drift locations were verified during the QA/QC process. In
the final, filtered drift-level data set the average time fished was X
minutes with a standard deviation (SD) of X. The average number of
observed anglers was X (SD=X), and average estimated depth was X (SD=X).

We define 108 areas of rocky habitat within California state waters from
the California/Oregon border to Point Conception. The two meter
resolution of the substrate shows the patchiness and heterogeneity of
the rocky substrate (Figure~\ref{fig-map2}). We characterize rocky
habitat using thresholds as determined by the CSMP
(\citet{CSUMB:2014:CSM}). While the location-specific data from the
fishing fleet is governed by confidentiality and cannot be displayed
here, 85\% of the fishing drifts were within 5 m of rocky habitat. The
recreational fishing fleet's targeting of rockfish species was verified
by the distributions of the distance from rocky habitat for each of the
six species. The distance from rocky habitat cutoff (retaining 90\% of
drifts encountering each species) was six meters for blue, China and
gopher rockfish, eight meters for vermilion rockfish, 14 meters for
black rockfish and 16 meters for brown rockfish. The percentage of
drifts and trips encountering the target species can be found in
Table~\ref{tbl-samplesize}.

Based on exploratory analyses and consideration of the available data,
we aggregated the areas of rocky habitat grouped into six regions to
ensure adequate sample sizes for developing indices of abundance
(Figure~\ref{fig-map}). While covering a small area (5\% of the rocky
habitat), the number of observed fishing drifts within state waters
around the Farallon Islands off the coast of San Francisco was high
enough to warrant keeping it as a separate area of rocky habitat. The
region defined from the California/Oregon border to San Francisco
encompasses 49\% of the total rocky habitat in state waters by area, but
only 12\% of the observed drifts (2,637). Each of the four remaining
regions of rocky habitat defined from San Francisco to Point Conception
contained an average of 12\% of the available habitat (Table X). The
CDFW estimated fishing effort by management district, which does not
exactly align with our areas of grouped reef habitat. Only considering
the fishing effort north of Point Conception, CDFW estimated an average
of 9\% of the CPFV trips occurred from the California/Oregon border
through Mendocino County, 38\% from Sonoma through San Mateo County, and
53\% from Santa Cruz to Point Conception.

The differences in latitudinal distribution of the six species is
apparent from the maps of percent of positive observations
(Figure~\ref{fig-percentpos}). The distribution of black rockfish tapers
off south of San Francisco, whereas percent of fishing drifts
encountering vermilion, gopher, and blue rockfish are higher south of
San Francisco. Brown rockfish is distributed across all of coastal
California, with slightly higher encounter rates south of San Francisco
and the percentage of drifts retaining China rockfish was low coastwide.
The average CPUE was highest for blue rockfish between San Francisco
south to Big Sur (Figure~\ref{fig-cpue}). The average CPUE for black
rockfish average was higher in the north, while gopher rockfish CPUE was
generally consistent across the coast, albeit slightly higher south of
Big Sur. China rockfish CPUE catch was typically low coastwide, with
slightly higher catch rates in the Farallon Island reefs.

The final aggregation of the reefs and total area within each region are
found in Table~\ref{tbl-reefareas}. The fraction of drifts retained for
the indices of abundance was high for all six species (80\% or greater),
indicating that fishing effort represented by these data occurred mainly
near areas of rocky habitat.

\hypertarget{indices-of-abundance-1}{%
\subsection{Indices of Abundance}\label{indices-of-abundance-1}}

Model selection using AIC resulted in all but three of the 24 indices of
relative abundance bineg modeled with a lognormal distribution for
positive observations. The trip-level indices for black, blue and gopher
rockfish were modeled using a gamma (refer to the supplementary material
for AIC scores). All of the covariates (year, reef, and wave) were
selected as main effects for both the binomial and positive models for
all species in the habitat-informed drift level index. For instances
where the wave. However, the difference in AIC relative to the best
model (delta-AIC) was less than ten so we chose to maintain the model
with year, reef and wave.

(LMK and I can put these in a table in the doc) The full model that
included the reef:year interaction was selected by AIC for all species
except for China rockfish. For China rockfish the positive binomial
model selected the interaction covariate, but the model without the
interaction was select for the positive lognormal model by an difference
in AIC of 22. However, in order to look at the effects of the
area-weighting on the index, we included the year:reef interaction in
the final model for China rockfish.

For both the drift-level and trip-level Stephens-MacCall filtered data,
year, county and wave were selected for black rockfish, blue rockfish,
gopher rockfish, and vermilion rockfish and the drift-level index for
brown rockfish. The model incorporate in year and county was selected
for the trip-level Stephens-MacCall filtered index for brown rockfish
and both Stephens-MacCall filtered indices for China rockfish.

In general, the larger increases and decreases in the indices were
similar among the four indices developed for each species
(Figure~\ref{fig-indices}). The generalized approach used in this paper
to create indices with comparable methods resulted in different results
for each species. The area-weighted indices are reflective of the total
available habitat and use all of the available high resolution habitat
and fishing drift data. However, differences among the four indices were
different for each species. The average CVs between the drift-level
area-weighted index and the drift-level habitat informed indices were
similar, as expected, since they both used the same data with the only
difference being the year:area interaction in the models
(Table~\ref{tbl-avgcv}). However, the average CV between drift-level
habitat-informed filtering and Stephens-MacCall filtering for the
drift-level data differed by species.

The area-weighting for black rockfish, a species distributed
predominantly north of Santa Cruz, California did have an effect on the
index for a number of years, most notably in 2013 where the
area-weighted estimate is lower than all three other
indices(Figure~\ref{fig-black-indices}). The effect of the
area-weighting is also apparent for black rockfish in 2005, 2007, and
2009.The average CV decreased from the trip-level index (0.671) to to
the area-weighted index (0.443). Interestingly, the average CV was
lowest overall for the drift-level Stephens-MacCall index (0.364) which
also modeled much smaller data with a high proportion of positive
catches of black rockfish (Table~\ref{tbl-samplesize}).

Blue rockfish is ubiquitous across the study area and was one of the two
species for which the index was weighted by the six regions of rocky
reef habitat. The area-weighted index differs from the other three in
2006 with an estimated higher relative abundance and in 2014 with an
estimated lower relative abundance. Even during the years from 2009 to
2012 when the estimated relative abundance was low for all of the
indices, there were differences among the four trends with the
drift-level habitat-informed index estimating the lowest relative
abundance.

All four indices for brown rockfish suggested differing trends, with
this species having the highest estimated error for both the trip-level
and drift-level Stephens-MacCall filtered data
(Figure~\ref{fig-brown-indices}). In ten of the years the area-weighted
index estimated a either the largest or smallest relative abundance
compared to the other indices. For brown rockfish the two
habitat-informed indices were more similar than the Stephens-MacCall
filtered data. The average CV for brown rockfish from the
Stephens-MacCall filtering was large (0.679) compared to the habitat
informed filtering (0.142).

China rockfish is the only species for which the trip-level
Stephens-MacCall filtered index had the lowest average coefficient of
variation that increased with the habitat-informed filtering
(Table~\ref{tbl-avgcv}). Although the trends among the four indices were
similar, this is the only species for which the highest error was
consistently estimated for both habitat-informed drift-level indices
(Figure~\ref{fig-china-indices}). China rockfish is one of the less
common species observed in the data with the highest average CPUE from
catches the Farallon Islands, which is an overall small percentage of
the total habitat (Table~\ref{tbl-reefareas}).

The observed trends for gopher rockfish were similar among all indices
and the trip-level Stephens-MacCall index had the highest average CV
(0.626) compared to the average CVs of less than two from all of the
other drift-level indices. China rockfish is the only species for which
the trip-level index had the lowest average coefficient of variation,
which increased with the habitat-informed filtering . For all other
species, the habitat-informed filtering resulted indices with a lower
average CV than the trip-level filtering.

The indices of relative abundance for vermilion rockfish were relatively
similar in trends across the time series
(Figure~\ref{fig-vermilion-indices}). Vermilion rockfish is the second
species for which all six areas of rocky reef habitat remained
dis-aggregated in the models. For vermilion rockfish, while the trends
are similar among all four indices, the effect of area-weighting dampens
the increase modeled from the habitat-informed drift level data from
2004-2006, where the area-weighting down-weighted the relative abundance
from the drift-level habitat informed index.

\hypertarget{allocation-of-yield}{%
\subsection{Allocation of Yield}\label{allocation-of-yield}}

\FloatBarrier

\hypertarget{discussion}{%
\section{Discussion}\label{discussion}}

Fishery-dependent indices of abundance will continue to be incorporated
in fisheries stock assessments. We demonstrated the effects of
subsetting fishery-dependent survey data to samples representing
effective effort at varying levels of data resolution. The estimated
indices of abundance illustrated the changes in trends and variance to
create a subset of samples representing the effective effort for a
target species, and how that selection affected the trends in indices of
relative abundance. The combination of fine-scale CPUE data coupled with
the available habitat data creates allows us to model an index of
relative density, rather than abundance. The fishery-dependent onboard
observer survey conducted by CDFW and Cal Poly is a benchmark
recreational fishery-dependent time series. The survey provides many
elements that would usually only be collected from research surveys,
including fishing locations, fishing depth, time fished, and speciated
catch and discard information, and currently has over a 20 year time
series.

We also demonstrated that the habitat-informed data filtering provides a
method to select samples with effective fishing reduces the subject
decision points required when filtering multispecies data by utilizing
known habitat characteristics. This also allows us to create an
area-weighted index that accounts for variable species density along the
coast. This not only addressed a key assumption of identifying effective
fishing effort for a multispecies fishery, but also appropriately
weights the sample data based on the known area of habitat.

The addition of habitat information a The Stephens-MacCall filtering
method has several subjective decision points, including which species
to include in the analysis, the threshold to determine which samples to
retain or remove, and the spatial extent of data to include. The
Stephens-MacCall filter is useful in identifying co-occurring or
non-occurring species, but it assumes all effort was exerted in pursuit
of a single target species. Stephens-MacCall filtering is most often
used for data collected at the trip-level in the absence of known
fishing locations. If more than one species or species complex was
targeted during a trip it can result in co-occurrence of species in the
trip-level catch that do not truly co-occur. This was clearly shown in
the differences between the trip-level Stephens-MacCall filtering and
the drift-level Stephens-MacCall filtering that reflects species
co-occurrence at a finer scale. If the fishing drifts covered small
enough areas the Stephens-MacCall filter at the drift-level inherently
contains information on habitat preferences and community structure.

The choice of a threshold value to use from the Stephen-MacCall method
has been a topic explored within stock assessments for both commercial
and recreational data (\citet{Dettloff:2021:ISA};
\citet{Cope:2015:DMS};\citet{DucharmeBarth:2018:IAG}). There is
currently no guidance on best practices for the decision points in the
Stephens-MacCall method that may lead to additional bias in data
selection. For instance, all of the observations in the onboard observer
survey are recorded by trained samplers who are assumed to correctly
identify species. With this assumption, we retained all of the samples
observing the target species regardless of the probability estimated
from the Stephens-MacCall model. The drift-level habitat informed data
retained a larger number of drifts than the drift-level Stephens-MacCall
filtered data, as a result of the majority of drifts occurring over hard
bottom habitat. However, one caveat of the rocky reef habitat data is
that there is currenlty only a binomial classification of hard and soft
substrate available, and we assume that all rocky habitat is suitable
habitat. We know from the variability in rugosity and relief displayed
in Figure (Figure~\ref{fig-map2}) that these characteristics can change
at small spatial scales. The \emph{Sebastes spp.} complex north of Point
Conception have differential hard bottom preferences, which have been
verified by visual surveys (\citet{Laidig:2009:DFH};
\citet{Anderson:2007:MHA}; \citet{Haggarty:2006:CIR} ) and from
discussions with experienced fishermen.

Based on the current practice of retaining the false positives within
the Stephens-MacCall method as described in the methods section, the
trip-level data are prone to overestimate fishing effort for the less
common species, and result in larger variances in the indices of
abundance. Looking at the number of trips selected between the
drift-level Stephens-MacCall filter and the habitat-informed filter, the
Stephens-MacCall filter (based on the retention of the false negatives)
may exclude too many samples that fished in the appropriate habitat, but
did not meet the probability threshold (Table~\ref{tbl-samplesize}).
Looking at the number of trips selected between the drift-level
Stephens-MacCall filter and the habitat-informed filter, the
Stephens-MacCall filter (based on the retention of the false negatives)
may exclude too many samples that fished in the appropriate habitat, but
did not meet the probability threshold (Table~\ref{tbl-samplesize}). The
Stephens-MacCall filter may be over-selecting samples where the species
was not observed if the target species is less common, e.g., China
rockfish, but has a strong positive co-occurrence with a more common
midwater, schooling species, e.g., blue rockfish. China rockfish in
particular have a heterogenous distribution with an affinity to high
relief habitat (\citet{Love:2002:RNP}). The Stephens-MacCall filter may
be over-selecting samples where the species was not observed if the
target species is less common, e.g., China rockfish, but has a strong
positive co-occurrence with a more ubiquitous species, e.g., blue
rockfish. For a ubiquitous species like vermilion rockfish, the
Stephens-MacCall drift-level data included 51\% fewer drifts than the
habitat-informed data, and for the less common China rockfish, 84\%
fewer total samples. The Stephens-MacCall method applied at the
drift-level provides insight into the fine-scale species associations,
but may also reflect targetting of species that are more common or
schooling. The integration of the habitat data with the onboard observer
fishing drift locations provides the most accurate information for
filtering the survey data. The differences between the drift-level
Stephens-MacCall filtered data and the habitat-informed filter
illustrate what may represent the habitat preference of individual
species.

Areas of rocky habitat that were well fished and never observed the
target species should be investigated to determine if the appropriate
habitat exists in that area, or if other factors such as historical
fishing pressure explain the lack of target species catch. The suite of
six species that we modeled in this paper is a concrete example of why
habitat is important and also varies among the species. The high
proportion of retained drifts across species when using habitat as a
data filter indicates that hate majority of drifts occurred over, or
very close to, rocky habitat. Both blue and black rockfish have high
affinity to rocky habitat, but occur higher off the bottom and are both
schooling species. It is not uncommon to have a fishing drift dominated
by blue rockfish in central California, or black rockfish further north.
However, the Stephen-MacCall approach does not account for this by
modeling presence/absence. Additional factors such as latitude could be
included in the logistic regression to inform the Stephens-MacCall
model.

The majority of groundfish species targeted by the CPFV fleet north of
Point Conception during the time period of this study all have high
associations to rocky habitat. In this case, the Stephens-MacCall method
can be considered a proxy for habitat when the species of interest has
known associations. This can be expanded in areas where trips are known
to target species of interest, but no habitat data are available the
proportion of trips encountering the target species could be used as a
proxy for habitat.

The differences observed in the indices of abundance and knowledge of
species-specific habitat preference will allow us to fine-tune these
indices on a species-specific basis. The characteristics and
classification of the rocky reef habitat into more specific substrate
types, e.g., boulder vs pinnacle, are currently only available for a
small fraction of the mapped area. Therefore, all areas of rocky
substrate are currently created equal. A number of video surveys have
shown habitat associations differ by species \citep{Love:2002:RNP}, and
the weights applied as available habitat may vary by species and be
lower than the weights used in this paper. Although we did not exclude
data based on the species' distributions from the indices developed
here, the habitat-informed filters also allow an analyst to subset the
data and exclude areas of rocky reef habitat outside of the species'
range. For instance, black rockfish have been observed as far south as
Point Conception, but their distribution tapers off south of Santa Cruz,
California.

Fishery-dependent indices of abundance undergo higher levels of scrutiny
during stock assessment reviews due to the nature of the data being
driven by angler behavior. Catch from the recreational CPFV fishery is
dependent on a number of factors including weather, distance from port,
the clientele preferences, angler experience and captain's knowledge.
These models also do not account for distance to the nearest port, which
has been shown to significantly impact the access to fish as well as
historical fishing pressure \citep{Miller:2014:SDH}. There are
additional key assumptions made when using the onboard observer data in
a stock assessment, including that fishing behavior remains the same
when samplers are not onboard the vessel.

Catch from the recreational CPFV fishery is dependent on a number of
factors including weather, distance from port, the clientele
preferences, angler experience and captain's knowledge. These models
also do not account for distance to the nearest port, which has been
shown to significantly impact the access to fish as well as historical
fishing pressure. Recent studies have identified the need to investigate
the assumptions and uncertainty in relative indices of abundance from
visual surveys \citep{Bacheler:2015:ERA, Campbell:2015:CRA} and
simulation studies \citep{Siegfried:2016:ISA}, and the same holds true
for fishery-dependent surveys like the onboard observer survey. To
address the potential bias in angling data for groundfish species,
Haggarty and King \citeyearpar{Haggarty:2006:CIR} conducted a SCUBA dive
survey followed by a research angling survey directly above the dive
plots and found a strictly proportional relationship between the density
estimated from the SCUBA survey and CPUE from the angling survey for
copper rockfish, a species whose habitat and depth distribution were
well covered by the survey. Further analyses are underway to explore the
fine-scale habitat characteristics to fine-tune the habitat informed
data selection methods. We also plan to explore changes in fishing
behavior related to management measures and and fisher behavior to
explain shifts among target species or how large recruitment events for
one species may affect the index of abundance for another species. While
not all of these factors can be controlled for, defining the samples
with effective effort will provide the most accurate index and
appropriate variance for stock assessments.

removed: However, they found no relationship between SCUBA dive survey
data and the angling survey for kelp greenling (\emph{Hexagrammos
decagrammus}), which the authors hypothesized was due to the greenling's
avoidance of the bait used.

removed: Further analyses are underway to explore the fine-scale habitat
characteristics that will allow the methods described in this paper to
be fine-tuned. We also plan to explore changes in fishing behavior
related to management measures and and fisher behavior to explain shifts
among target species or how large recruitment events for one species may
affect the index of abundance for another species. removed: This does
not hold for areas where multiple species complexes are targeted on same
trip, e.g, a multi-day trip may target large pelagic species and once
trip limits are reached, the trip may focus on a secondary target, which
is the case for the California CPFV fleet fishing south of Point
Conception.

removed: An additional source of bias in fishery-dependent data is the
change in regulation over time. These can be bag limits, minimum size
restrictions, and area closures that the change of available habitat.
Depth restrictions have also been in place for the recreational fleet
since the early 2000s, which were relaxed in 2017 and was the reason we
constrained the years modeled for this study.

removed: Versions of the drift-level habitat-informed indices were
approved by the Pacific Fisheries Management Council's Science and
Statistical Committee for use in the 2013 stock assessments and have
been used in the stock assessment process since. Comparisons should not
be drawn between the indices presented here and the stock assessment
documents as the indices in this paper were simplified to develop direct
comparisons among methods. When filtering and modeling the onboard
observer data for a stock assessment, additional filtering steps would
be taken, such as excluding areas where species are rare, e.g., south of
Santa Cruz for black rockfish, inclusion of depth as a covariate in the
index of abundance, and an exploration of alternative error
distributions.

removed: Another example is closures and retraction of the available
habitat open to fishing. California developed a network of Marine
Protected Areas (MPAs) in 2007, that reduced the available rocky reef
habitat to the recreational fleet by approximately 23\% in state waters
north of Point Conception.

\hypertarget{acknowledgements}{%
\section{Acknowledgements}\label{acknowledgements}}

We thank the following reviewers for comments that improved the
manuscript. CDFW for collection of the onboard observer data Cal Poly
for the collection of the supplemental data

Data attribution: CDFW acquires data from its own fisheries management
activities and from mandatory reporting requirements on the commercial
and recreational fishery pursuant to the Fish and Game Code and the
California Code of Regulations. These data are constantly being updated,
and data sets are constantly modified. CDFW may provide data upon
request, but, unless otherwise stated, does not endorse any particular
analytical methods, interpretations, or conclusions based upon the data
it provides.

\hypertarget{tables}{%
\section{Tables}\label{tables}}

\FloatBarrier

\hypertarget{tbl-reefareas}{}
\begin{table}
\caption{\label{tbl-reefareas}Area of rocky habitat in state waters aggregated to the levels modelled
for each species. The merged cells for each species indicate which areas
of rocky habitat were aggregated to ensure appropriate samples sizes to
explore an area-weighted index. }\tabularnewline

\centering
\resizebox{\linewidth}{!}{
\begin{tabular}{>{\raggedright\arraybackslash}p{1.6in}|>{\raggedleft\arraybackslash}p{.8in}|>{\raggedleft\arraybackslash}p{.8in}|>{\raggedleft\arraybackslash}p{.8in}|>{\raggedleft\arraybackslash}p{.8in}|>{\raggedleft\arraybackslash}p{.8in}}
\toprule
Rocky Reef Desginations & Blue rockfish \& Vermilion rockfish & Black rockfish & Brown rockfish & China rockfish & Gopher rockfish\\
\midrule
California border to San Francisco & 439.546 & 439.546 & 439.546 &  & \\
\cmidrule{1-4}
San Francisco to Santa Cruz & 108.424 & 108.424 &  & \multirow{-2}{.8in}{\raggedleft\arraybackslash 547.970} & \\
\cmidrule{1-3}
\cmidrule{5-5}
Farallon Islands & 50.252 &  & \multirow{-2}{.8in}{\raggedleft\arraybackslash 498.967} & 50.252 & \\
\cmidrule{1-2}
\cmidrule{4-5}
Moss Landing to Big Sur & 137.603 &  &  & 137.603 & \multirow{-4}{.8in}{\raggedleft\arraybackslash 735.825}\\
\cmidrule{1-2}
\cmidrule{5-6}
Big Sur to Morro Bay & 90.424 &  & \multirow{-2}{.8in}{\raggedleft\arraybackslash 228.027} &  & 90.424\\
\cmidrule{1-2}
\cmidrule{4-4}
\cmidrule{6-6}
Morro Bay to Point Conception & 112.264 & \multirow{-4}{.8in}{\raggedleft\arraybackslash 390.543} & 112.264 & \multirow{-2}{.8in}{\raggedleft\arraybackslash 202.688} & 112.264\\
\bottomrule
\end{tabular}}
\end{table}

\FloatBarrier

\hypertarget{tbl-samplesize}{}
\begin{table}
\caption{\label{tbl-samplesize}The number of samples retained after filtering to create the index of
abundance with the percent of samples that caught the species in
parentheses. }\tabularnewline

\centering
\resizebox{\linewidth}{!}{
\begin{tabular}{>{\raggedright\arraybackslash}p{1.6in}lll}
\toprule
\multicolumn{1}{c}{ } & \multicolumn{2}{c}{Drift-level} & \multicolumn{1}{c}{Trip-level} \\
\cmidrule(l{3pt}r{3pt}){2-3} \cmidrule(l{3pt}r{3pt}){4-4}
Species & Habitat-informed & Stephens-MacCall filtered & Stephens-MacCall filtered\\
\midrule
Black Rockfish & 16306 (16\%) & 3038 (30\%) & 706 (68\%)\\
Blue Rockfish & 15283 (44\%) & 7490 (60\%) & 1813 (91\%)\\
Brown Rockfish & 15736 (16\%) & 2740 (31\%) & 806 (62\%)\\
China Rockfish & 14865 (8\%) & 1331 (22\%) & 798 (57\%)\\
Gopher Rockfish & 14476 (31\%) & 5088 (45\%) & 1449 (81\%)\\
\addlinespace
Vermilion Rockfish & 14713 (30\%) & 5040 (45\%) & 1627 (85\%)\\
\bottomrule
\end{tabular}}
\end{table}

\FloatBarrier

\hypertarget{tbl-avgcv}{}
\begin{table}
\caption{\label{tbl-avgcv}The average Coefficient of Variation (CV) for each index of abundance,
where SM-filtered is the Stephens-MacCall filtering. }\tabularnewline

\centering
\resizebox{\linewidth}{!}{
\begin{tabular}{lrrrr}
\toprule
\multicolumn{1}{c}{ } & \multicolumn{3}{c}{Drift-level} & \multicolumn{1}{c}{Trip-level} \\
\cmidrule(l{3pt}r{3pt}){2-4} \cmidrule(l{3pt}r{3pt}){5-5}
Species & Area-weighted & Habitat-informed & Stephens-MacCall filtered & Stephens-MacCall filtered\\
\midrule
Black rockfish & 0.4426091 & 0.4493133 & 0.7641099 & 0.8495448\\
Blue rockfish & 0.1343866 & 0.1415416 & 0.1610735 & 0.3324914\\
Brown rockfish & 0.2415686 & 0.2399299 & 0.8652880 & 0.9161881\\
China rockfish & 0.3196653 & 0.3011640 & 0.4481187 & 0.2087114\\
Gopher rockfish & 0.1785421 & 0.1831132 & 0.2562205 & 0.2535190\\
\addlinespace
Vermilion rockfish & 0.1519120 & 0.1781884 & 0.4224451 & 0.5087889\\
\bottomrule
\end{tabular}}
\end{table}

\FloatBarrier

\hypertarget{figures}{%
\section{Figures}\label{figures}}

\begin{figure}

{\centering \includegraphics{figures/map_2.jpg}

}

\caption{\label{fig-map2}A example of the high resolution bathymetric
data and components of bathymetry and rugosity used to define rough
versus smooth substrate (where hard substrate is denoted by 1). The far
right panel displays the hard substrate with the added 5 m buffer to
represent the rocky reef habitat.}

\end{figure}

\begin{figure}

{\centering \includegraphics{figures/map.jpg}

}

\caption{\label{fig-map}A maps of California state waters north of Point
Conception colored by the aggregated areas of rocky reef habitat,
including inset A depicting the rocky reef habitat in relation to 3 nm
state water boundary state waters and inset B showing the high
resolution rocky habitat in the area.}

\end{figure}

\begin{figure}

{\centering \includegraphics{figures/percentpositives_map.jpg}

}

\caption{\label{fig-percentpos}The percent of drifts that retained the
target species, within grouped areas of rocky habitat over all years of
the time series. The grey dashed lines represent the aggregated rocky
habitat used to develop an index of abundance.}

\end{figure}

\begin{figure}

{\centering \includegraphics{figures/CPUE_map.jpg}

}

\caption{\label{fig-cpue}The average CPUE across all years of the time
series for each of the six species. The grey dashed lines represent the
aggregated rocky habitat used to develop an index of abundance.}

\end{figure}

\begin{figure}

\begin{minipage}[t]{0.50\linewidth}

{\centering 

\raisebox{-\height}{

\includegraphics[width=2.7in,height=\textheight]{figures/black_trip_sm.png}

}

}

\subcaption{\label{fig-black-tripsm}Black rockfish trip-level}
\end{minipage}%
%
\begin{minipage}[t]{0.50\linewidth}

{\centering 

\raisebox{-\height}{

\includegraphics[width=2.7in,height=\textheight]{figures/brown_trip_sm.png}

}

}

\subcaption{\label{fig-brown-tripsm}Brown rockfish trip-level}
\end{minipage}%

\caption{\label{fig-sm}Examples of the species coefficients and 95\%
confidence intervals for the Stephens-MacCall filtering for black
rockfish (a) and brown rockfish (b) in the trip-level data. A positive
coefficient indicates a species is associated with the target species
and a negative coefficient indicates the species is not associated with
the target species.}

\end{figure}

\begin{figure}

\begin{minipage}[t]{0.50\linewidth}

{\centering 

\raisebox{-\height}{

\includegraphics[width=3in,height=\textheight]{figures/black_indices.png}

}

}

\subcaption{\label{fig-black-indices}Black rockfish}
\end{minipage}%
%
\begin{minipage}[t]{0.50\linewidth}

{\centering 

\raisebox{-\height}{

\includegraphics[width=3in,height=\textheight]{figures/blue_indices.png}

}

}

\subcaption{\label{fig-blue-indices}Blue rockfish}
\end{minipage}%
\newline
\begin{minipage}[t]{0.50\linewidth}

{\centering 

\raisebox{-\height}{

\includegraphics[width=3in,height=\textheight]{figures/brown_indices.png}

}

}

\subcaption{\label{fig-brown-indices}Brown rockfish}
\end{minipage}%
%
\begin{minipage}[t]{0.50\linewidth}

{\centering 

\raisebox{-\height}{

\includegraphics[width=3in,height=\textheight]{figures/china_indices.png}

}

}

\subcaption{\label{fig-china-indices}China rockfish}
\end{minipage}%
\newline
\begin{minipage}[t]{0.50\linewidth}

{\centering 

\raisebox{-\height}{

\includegraphics[width=3in,height=\textheight]{figures/gopher_indices.png}

}

}

\subcaption{\label{fig-gopher-indices}Gopher rockfish}
\end{minipage}%
%
\begin{minipage}[t]{0.50\linewidth}

{\centering 

\raisebox{-\height}{

\includegraphics[width=3in,height=\textheight]{figures/vermilion_indices.png}

}

}

\subcaption{\label{fig-vermilion-indices}Vermilion rockfish}
\end{minipage}%

\caption{\label{fig-indices}Indices of abundance and 95\% confidence
intervals for the different filtering strategies, each scaled to its
mean, for the six species.}

\end{figure}

\FloatBarrier


  \bibliography{bibliography.bib}


\end{document}
