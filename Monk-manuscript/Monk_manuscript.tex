% Options for packages loaded elsewhere
\PassOptionsToPackage{unicode}{hyperref}
\PassOptionsToPackage{hyphens}{url}
\PassOptionsToPackage{dvipsnames,svgnames,x11names}{xcolor}
%
\documentclass[
  12pt,
  authoryear,
  preprint,
  3p]{elsarticle}

\usepackage{amsmath,amssymb}
\usepackage{lmodern}
\usepackage{iftex}
\ifPDFTeX
  \usepackage[T1]{fontenc}
  \usepackage[utf8]{inputenc}
  \usepackage{textcomp} % provide euro and other symbols
\else % if luatex or xetex
  \usepackage{unicode-math}
  \defaultfontfeatures{Scale=MatchLowercase}
  \defaultfontfeatures[\rmfamily]{Ligatures=TeX,Scale=1}
\fi
% Use upquote if available, for straight quotes in verbatim environments
\IfFileExists{upquote.sty}{\usepackage{upquote}}{}
\IfFileExists{microtype.sty}{% use microtype if available
  \usepackage[]{microtype}
  \UseMicrotypeSet[protrusion]{basicmath} % disable protrusion for tt fonts
}{}
\makeatletter
\@ifundefined{KOMAClassName}{% if non-KOMA class
  \IfFileExists{parskip.sty}{%
    \usepackage{parskip}
  }{% else
    \setlength{\parindent}{0pt}
    \setlength{\parskip}{6pt plus 2pt minus 1pt}}
}{% if KOMA class
  \KOMAoptions{parskip=half}}
\makeatother
\usepackage{xcolor}
\setlength{\emergencystretch}{3em} % prevent overfull lines
\setcounter{secnumdepth}{5}
% Make \paragraph and \subparagraph free-standing
\ifx\paragraph\undefined\else
  \let\oldparagraph\paragraph
  \renewcommand{\paragraph}[1]{\oldparagraph{#1}\mbox{}}
\fi
\ifx\subparagraph\undefined\else
  \let\oldsubparagraph\subparagraph
  \renewcommand{\subparagraph}[1]{\oldsubparagraph{#1}\mbox{}}
\fi


\providecommand{\tightlist}{%
  \setlength{\itemsep}{0pt}\setlength{\parskip}{0pt}}\usepackage{longtable,booktabs,array}
\usepackage{calc} % for calculating minipage widths
% Correct order of tables after \paragraph or \subparagraph
\usepackage{etoolbox}
\makeatletter
\patchcmd\longtable{\par}{\if@noskipsec\mbox{}\fi\par}{}{}
\makeatother
% Allow footnotes in longtable head/foot
\IfFileExists{footnotehyper.sty}{\usepackage{footnotehyper}}{\usepackage{footnote}}
\makesavenoteenv{longtable}
\usepackage{graphicx}
\makeatletter
\def\maxwidth{\ifdim\Gin@nat@width>\linewidth\linewidth\else\Gin@nat@width\fi}
\def\maxheight{\ifdim\Gin@nat@height>\textheight\textheight\else\Gin@nat@height\fi}
\makeatother
% Scale images if necessary, so that they will not overflow the page
% margins by default, and it is still possible to overwrite the defaults
% using explicit options in \includegraphics[width, height, ...]{}
\setkeys{Gin}{width=\maxwidth,height=\maxheight,keepaspectratio}
% Set default figure placement to htbp
\makeatletter
\def\fps@figure{htbp}
\makeatother

\usepackage{booktabs}
\usepackage{longtable}
\usepackage{array}
\usepackage{multirow}
\usepackage{wrapfig}
\usepackage{float}
\usepackage{colortbl}
\usepackage{pdflscape}
\usepackage{tabu}
\usepackage{threeparttable}
\usepackage{threeparttablex}
\usepackage[normalem]{ulem}
\usepackage{makecell}
\usepackage{xcolor}
\usepackage{placeins}
\usepackage{setspace}
\usepackage{lineno}
\onehalfspacing
\linespread{2}
\linenumbers
\makeatletter
\makeatother
\makeatletter
\makeatother
\makeatletter
\@ifpackageloaded{caption}{}{\usepackage{caption}}
\AtBeginDocument{%
\ifdefined\contentsname
  \renewcommand*\contentsname{Table of contents}
\else
  \newcommand\contentsname{Table of contents}
\fi
\ifdefined\listfigurename
  \renewcommand*\listfigurename{List of Figures}
\else
  \newcommand\listfigurename{List of Figures}
\fi
\ifdefined\listtablename
  \renewcommand*\listtablename{List of Tables}
\else
  \newcommand\listtablename{List of Tables}
\fi
\ifdefined\figurename
  \renewcommand*\figurename{Figure}
\else
  \newcommand\figurename{Figure}
\fi
\ifdefined\tablename
  \renewcommand*\tablename{Table}
\else
  \newcommand\tablename{Table}
\fi
}
\@ifpackageloaded{float}{}{\usepackage{float}}
\floatstyle{ruled}
\@ifundefined{c@chapter}{\newfloat{codelisting}{h}{lop}}{\newfloat{codelisting}{h}{lop}[chapter]}
\floatname{codelisting}{Listing}
\newcommand*\listoflistings{\listof{codelisting}{List of Listings}}
\makeatother
\makeatletter
\@ifpackageloaded{caption}{}{\usepackage{caption}}
\@ifpackageloaded{subcaption}{}{\usepackage{subcaption}}
\makeatother
\makeatletter
\@ifpackageloaded{tcolorbox}{}{\usepackage[many]{tcolorbox}}
\makeatother
\makeatletter
\@ifundefined{shadecolor}{\definecolor{shadecolor}{rgb}{.97, .97, .97}}
\makeatother
\makeatletter
\makeatother
\journal{Fisheries Research}
\ifLuaTeX
  \usepackage{selnolig}  % disable illegal ligatures
\fi
\usepackage[]{natbib}
\bibliographystyle{elsarticle-harv}
\IfFileExists{bookmark.sty}{\usepackage{bookmark}}{\usepackage{hyperref}}
\IfFileExists{xurl.sty}{\usepackage{xurl}}{} % add URL line breaks if available
\urlstyle{same} % disable monospaced font for URLs
\hypersetup{
  pdftitle={Methods to utilize known habitat to filter data for indices of abundance from a recreational fishery survey in California},
  pdfauthor={Melissa Hedges Monk; Rebecca R. Miller; Grant Waltz; Dean Wendt},
  pdfkeywords={fisheries dependent data, habitat
association, groundfish, index of abundance},
  colorlinks=true,
  linkcolor={blue},
  filecolor={Maroon},
  citecolor={Blue},
  urlcolor={Blue},
  pdfcreator={LaTeX via pandoc}}

\setlength{\parindent}{6pt}
\begin{document}

\begin{frontmatter}
\title{Methods to utilize known habitat to filter data for indices of
abundance from a recreational fishery survey in California}
\author[1]{Melissa Hedges Monk%
\corref{cor1}%
\fnref{fn1}}
 \ead{melissa.monk@noaa.gov} 
\author[2]{Rebecca R. Miller%
%
}
 \ead{rebecca.miller@noaa.gov} 
\author[33]{Grant Waltz%
%
}
 \ead{cat@example.com} 
\author[3]{Dean Wendt%
%
}
 \ead{cat@example.com} 

\affiliation[1]{organization={Southwest Fisheries Science
Center}, addressline={110 McAllister Way}, city={Santa
Cruz}, country={}, postcode={95060}}

\affiliation[2]{organization={University of California Santa
Cruz}, addressline={Street Address}, city={Santa
Cruz}, country={}, postcode={95060}}

\affiliation[3]{organization={California Polytechnic State
University}, addressline={Street Address}, city={San Luis
Obispo}, country={}, postcode={93407}}


\cortext[cor1]{Corresponding author}
\fntext[fn1]{This is the first author footnote.}



        
\begin{abstract}
This is the abstract.
\end{abstract}





\begin{keyword}
    fisheries dependent data \sep habitat
association \sep groundfish \sep 
    index of abundance
\end{keyword}
\end{frontmatter}\ifdefined\Shaded\renewenvironment{Shaded}{\begin{tcolorbox}[frame hidden, sharp corners, interior hidden, borderline west={3pt}{0pt}{shadecolor}, enhanced, breakable, boxrule=0pt]}{\end{tcolorbox}}\fi

\hypertarget{introduction}{%
\section{Introduction}\label{introduction}}

Melissa will refocus the introduction - I keep changing my mind.

Integrated fisheries stock assessment models utilize a variety of data
sources to develop the most complete picture of the stock and current
status in relation to management thresholds.

Catch per unit effort is one

More often(TYPICALLY??), an index of abundance is used as a relative
measure of the population and requires a time series to inform the stock
assessment model. An index of relative abundance assumes that changes in
the index are proportional to changes of abundance in the population
\citep{Harley:2001:CUE}. Fishery-independent data collected from
standardized survey designs provide a more unbiased estimation of the
trend in a fisheries population. However, fishery-independent surveys
are costly, labor intensive and often require a long time series to be
considered informative in fisheries stock assessments. In an ideal
situation, both fishery-dependent and fishery-independent surveys would
be used to inform the stock assessment model. However, there are cases
when only fishery-dependent data are available and the caveats of each
data stream must be considered (cite assessments). Fishery-dependent
data are collected directly from the the fishery and are less costly
than (than what????) the whose operations are not constrained by
sampling designs, but dependent on the behaviors of the captain and
vessel and, in the case of recreational trips, customer preference.

Fishery-dependent surveys sample the fishing fleets and are subject to
potential sampling biases. The sampling is dependent on the fishing
boat's behavior, which is to maximize catch. Sampling of the fishing
fleet is often opportunistic based on the availability of samplers and
the availability of trips to sample. Sampling the fisheries can also be
constrained to the current regulations, which may prohibit the retention
of a species or fishing at certain depths, i.e., California Department
of Fish and Wildlife has varying spatial and temporal depth and season
closures implemented through six management regions. There is also a
network of Marine Protected Areas (MPAs) designated from 2007-2012 that
prohibit recreational fishing, and are therefore areas no longer sampled
by the recreational fishing fleet.

\emph{Catch per unit effort (CPUE) is a common metric available from
fishery-dependent surveys \citep{Maunder:2004:SCE}.}

A common characteristic of ecological data is a high proportion of zero
observations across samples and the question as to whether the sampling
occurred within the species' habitat and the species was not observed or
if the sampling occurred outside of the species' habitat (structural
zeroes). \textbf{you might mention someplace in introduction that
nearshore rockfish are strongly associated with rocky habitat-- perhaps
here. Also, consider this sentence/paragraph as the opening paragraph??}
Fisheries survey data are often subset to exclude structural zeroes
using the Stephens-MacCall method \citeyearpar{Stephens:2004:MAS}, which
models the probability of observing the target species given the other
the presence/absence of other species. However, the onboard observer
survey collected location-specific information on each fish encountered.
To subset the onboard observer survey data and exclude structural
zeroes,we used the positive catch locations (I DON''T TOTALLY UNDERSTAND
THIS SENTENCE- only postiive catches are included?) this sentence seems
like it is in the wrong place? more of a methods?) as a proxy for
suitable habitat.

To explore the changes in data filtering related to structural zeroes,
we utilized high resolution fishery sampling and and bathymetry data, we
evaluated data from a recreational party boat onboard observer survey,
which collects location- and species-specific CPUE information from the
commercial passenger fishing vessel (CPFV; also know as party boat)
fleet \citep{Monk:2014:DRD}. The data were collected at the level of a
fishing drift and fine-scale habitat data are available for a large
fraction of California state waters. Paired with recently available
high-resolution bathymetry data provided an opportunity to overlay each
individual fishing drift onto known habitat type (hard vs.~soft
substrate), and has been the method utilized for stock assessments since
2015. To explore how data selection methods and the resulting indices
would change if the data were only available at a courser resolution, we
used the same data set to develop standardized indices of relative
abundance based on three different data filtering methods. We applied
these methods across six nearshore rockfish species with different life
histories,habitat preferences and commonness in the data.

The three data treatment methods included filtering the drift-level data
based on known location, i.e., the status quo \textbf{(what is status
quo? StevensMcCall?}), treating the drift-level data as if the location
of the drifts were not available, and lastly, an aggregating of the
catches at the drift-level data to a trip. In addition, for the model
filtered based on known rocky reef habitat, we weighted the index by
area of habitat within pre-defined regions. For the two cases where we
removed the location information, we filtered the data using the
Stephens-MacCall method.

\emph{MOVE TO INTRODUCTION: It is not often the case where
high-resolution habitat and fishing location information are both
available, and for many fishery-dependent surveys an analyst will have
to determine which subset of the data to use based on available
information. The onboard observer data provide an opportunity to explore
what information we gain from explicit knowledge of fishing locations.}

The Stephens-MacCall \citeyearpar{Stephens:2004:MAS} filtering approach
was used to predict the probability of encountering the target species,
based on the species composition of the catch in a given trip. The
method uses presence/absence data within a logistic regression to
identify the probability of encountering a target species given the
presence or absence of other predictor species. This method is commonly
used to filter data that were collected dockside after a vessel returned
to port or when location data are not provided.

Prior to 2013, these data had not been used to develop an index of
abundance for West Coast fisheries stock assessments. That was partially
due to the availability of data and at that point in time not many full
stock assessments had been conducted for the nearshore groundfish
species. These data were first used without the habitat data in a suite
of 2013 data moderate stock assessments \citep{Cope:2015:DMS}, where
data were filtered using positive species observations and an alpha hull
method commonly used to estimate species home range sizes and
distribution models for terrestrial species
\citep{Burgman:2003:BSR, Meyer:2017:CHM}

\emph{Include a paragaph in the introduction that briefly summarizes the
purpose of the S-M method, i.e.~using species composition of the catch
to identify effective fishing effort for the target species, and cite it
there. That way, you can mention it in the methods and the reader will
already be familiar with it.}

\emph{cut from another place: The onboard observer survey data provide a
high-resolution of catch, effort and the ability to map the fishing
drifts to fine-scale habitat data. This paper explores methodological
differences in data treatment to see what we gain by having the
high-resolution habitat data and using that as a mechanism to filter out
trips that are not targeting the species of interest}

\emph{This paper explores methodological differences in data treatment
to determine changes in trends in indices and the associated error among
three alternative assumptions and data filtering strategies. All of the
methods described below started with the same subset of drifts from the
onboard observer survey data, restricted to state waters and the years
2004-2016. In the case of application to stock assessments, all
potential data are explored, which may be why trends in indices differ
in this paper than what has previously been published in stock
assessments. Since the most recent stock assessments in 2021, the data
have undergone a major quality assurance effort by the authors.}

\hypertarget{methods}{%
\section{Methods}\label{methods}}

We developed indicies of abundance for six species or species pairs of
rockfish (\emph{Sebastes spp.}) that are of management interest on the
U.S. West Coast: black rockfish (\emph{S. melanops}), the blue and
deacon rockfish (\emph{S. mystinus}/\emph{S. diaconus}), brown rockfish
(\emph{S. auriculatus}), China rockfish (\emph{S. nebulosus}), gopher
rockfish (\emph{S. carnatus}), and the vermilion and sunset rockfish
(\emph{S. miniatus}/\emph{S. crocotulus}). The two cryptic species pairs
(blue/deacon and sunset/vermilion rockfish) are genetically
identifiable, but not separable within the onboard observer survey time
series. These six species all have different latitudinal distributions,
exploitation histories, and habitat and depth
preferences\citep{Love:2002:RNP}.

\hypertarget{survey-data-and-habitat-based-filtering}{%
\subsection{Survey Data and Habitat-based
Filtering}\label{survey-data-and-habitat-based-filtering}}

The California Department of Fish and Wildlife (CDFW) began a
fishery-dependent onboard observer survey of the Commercial Passenger
Fishing Vessel (CPFV or party/charter boat) fleet in 1999. In 2004, the
survey became part of the CDFW's California Recreational Fisheries
Survey (CFRS, \emph{add year and cite website}) that includes additional
surveys to quantify catch and effort by the recreational fleet. In
response to a request from the fishing industry, the California
Polytechnic State University Institute of Marine Science, San Luis
Obispo (Cal Poly) began a supplemental onboard observer survey in 2001
of the CPFV fleet based in Port Avila and Port San Luis along the
Central Coast {[}\#fig-map{]}. Both the CDFW and the Cal Poly onboard
observer surveys continue through present day; however, due to both
spatial and temporal recreational regulation changes we limited the data
for this research to the years 2004 to 2016. Between 1999 and 2003, the
recreational regulations evolved from no restriction on the number of
lines or hooks an angler could deploy to a one line and two-hook
maximum, as well as implementation of depth restrictions. Subsequent
management allowed a relaxation of depth restrictions beginning in 2017,
potentially shifting fishing effort relative to the 2004-2016 period
\citep{Monk:2021:SVR}.

While only a small portion of the total CPFV trips taken are sampled as
part of the onboard observer survey, the onboard observer survey
collects a large amount of data during each trip. During each trip the
sampler records information for each fishing drift, defined as a period
starting when the captain announced ``lines down'' to when the captain
instructs anglers to reel their lines up. Just prior to the start of
each fishing drift, the sampler selected a subset of anglers to observe,
at a maximum of 15 anglers per drift. The sampler records all fish
encountered (retained and discarded) by the subset of anglers as a
group, i.e., catch cannot be attributed to an individual angler.
Samplers also record the start and end times of a drift, location of the
fishing drift (start latitude/longitude and for most drifts, end
latitude/longitude), and minimum and maximum bottom depth. Fish
encountered by the group of observed anglers are recorded as either
retained or discarded. This provides information on the catch (count of
each species) and effort (time and number of anglers fished) during each
fishing drift. While both surveys include records of discarded fish, we
only used the retained catch in these analyses. Discarded fish can often
represent a different size structure than retained fish, either due to
size limits or angler preference, or represent fish encountered during a
temporal or spatial closure.

The SWFSC developed a relational database for the CDFW onboard survey
from 1999-2010\citeyearpar{Monk:2014:DRD} that has been updated
annually. The Cal Poly data are also provided to the SWFSC annually. All
data were checked for potential errors at the drift-level by SWFSC
staff.

The CPFV data included only areas north of Point Conception
(\(34^\circ 27^\prime N\)) due to gaps in habitat coverage further
south. To further remove drifts that may not accurately define a
successful fishing drift or represent data errors, the upper and lower
1\% of the recorded time fished and recorded observed anglers were
removed. Given that the fishery was closed deeper than 40 fathoms for
the entire time period from 2004-2016, we filtered the data to retain
99\% of all drifts based on average drift depth. We calculated average
depth from the recorded minimum and maximum depths when available or the
imputed minimum and maximum depth from the bathymetry layer described in
the next paragraph. A depth cutoff slightly deeper than the maximum
allowed is reasonable given the variability in habitat fished and all
retained drifts occurred within California state waters (up to 3 nm from
shore).

High resolution seafloor mapping data allowed us to map each drift from
the onboard observer surveys with predicted habitat (referred throughout
the paper as the drift-level, habitat-informed data). We utilized the
bathymetry and backscatter data collected by the California Seafloor
Mapping Program (CSMP) \citep{Golden:2013:CSW}. The CSMP mapped
California state waters at a 2 m resolution north of of Point Conception
to the California-Oregon border. A total of 137 CSMP substrate blocks
that ranged in size from 16 \(km^2\) to more than 400 \(km^2\) were
mosaicked together by authors. Rough and smooth substrates were
identified by CSMP using two rugosity indices, surface:planar area, and
vector ruggedness measure (VRM) of the bathymetric digital elevation
model {[}\#fig-map2{]}. The CSMP set a varying VRM threshold for each of
the substrate blocks, removed any artifacts, and is considered a
conservative estimate of rough habitat.

The 137 CSMP substrate raster blocks were then mosaicked together by
authors, and converted the pixels designated as rough habitat (rocky
habitat proxy) from a raster format to polygons, and calculated a 5 m
buffer around the rough habitat polygon to allow for any small errors in
positional accuracy using ArcMap 10.7 (ESRI citation). The area of each
reef polygon was calculated, and those reefs greater than or equal to
100 \(m^2\) were included. Contiguous polygons identified as rocky
substrate were defined as a singular rocky reef, regardless of size. The
area of rocky habitat for this paper was calculated to exclude portions
of the reef that extended outside of California state waters (further
than 3 nm from shore). The mapped area does not include very shallow
areas close to shore, which extend approximately 200-500 m from the
shoreline. Fishing by the CPFV fleet is limited in these waters due to
shallow depths and kelp beds. We assigned fishing drifts to reefs based
on the recorded start location of a drift, given that the end locations
of drifts were not always recorded. The distance from the recorded drift
start location to the nearest rocky habitat was calculated in meters.
For each target species, we calculated the cumulative distribution of
distance to rocky reef for drifts that retained the target species and
used a distance cutoff of 90\% for each species. To illustrate the
similarities and differences among the six species, we plotted the
percent of fishing drifts within an aggregated region that where the
species was present and retained. To show the differences in the general
commonness or rarity of the species we calculated the average CPUE,
before standardization, for each species and aggregated area. We also
downloaded the effort estimates for the CPFV trips from RecFIN to
compare the the the area of rocky habitat with the effort in each region
as well as the distribution of observed trips.

\hypertarget{stephens-maccall-data-filtering}{%
\subsection{Stephens-MacCall Data
Filtering}\label{stephens-maccall-data-filtering}}

We applied the Stephens-MacCall method to both the drift-level data and
the trip-level data \citeyearpar{Stephens:2004:MAS}. For the drift-level
data we removed all location and depth identifiers for a drift and kept
the county of landing as a spatial identifier. To construct a data set
that mimicked trip-level data, we took the drift-level data, aggregated
the observed retained catch within a trip, and kept the county of
landing as a spatial identifier. We then compared results using two
levels of aggregation (catch rates by drift and trip) to illustrate the
impact of having less spatially-explicit data on both data filtering and
the resulting indices of abundance.

Prior to any filtering a total of 19,425 drifts that aggregated to 2,270
trips were available for the analyses. The number of initial samples
used for the Stephens-MacCall filtering method were higher than the
habitat-informed data described in the previous section because retained
drifts with missing locations (latitude/longitude).

Before applying the Stephens-MacCall method, we identified a suite of
potentially informative predictor species for each of the six target
species. Species that never co-occurred with the target species and
those present in fewer than 1\% of all drifts and 3\% of all trips were
removed to reduce the number of species to those that were informative.
A lower threshold of 1\% was selected for the drift-level data due to
the change in magnitude of the number of samples when using drifts vs
trips.

The remaining species all co-occurred with the target species in at
least one trip and were retained for the Stephens-MacCall logistic
regression. Coefficients from the Stephens-MacCall analysis (a binomial
generalized linear model) were positive for species that are more likely
to co-occur with the target species, and negative for species that were
less likely to be caught with target species. The intercept represented
the probability of observing only the target species in a sample. We
also calculated the 95\% confidence interval for each coefficient.

Stephens and MacCall proposed filtering (excluding) samples from index
standardization based on a criterion of balancing the number of false
positives and false negatives from the predicted probability of
encounter. False positives (FP) are trips that are predicted to
encounter the target species based on the species composition of the
catch, but did not. False negatives (FN) are trips that were not
predicted to encounter the target species, given the catch composition,
but caught at least one target species. Stephens and MacCall recommended
a threshold where the false negatives and false positives are equally
balanced, however, this threshold does not have any biological relevance
and for this particular data set where trained samplers identify all
fish. We assumed that if the target species was encountered, the vessel
fished in appropriate habitat.

Of interest for the index of abundance was the elimination of trips that
had a low probability of catching the target species given the other
species caught on the trip. Therefore, we retained all of the trips that
caught the target species and those trips that did not catch the target
species, but had a probability higher than the threshold balancing the
false negatives and false positives. This practice has commonly been
used in recent stock assessments of rockfish on the West Coast.

\hypertarget{indices-of-abundance}{%
\subsection{Indices of Abundance}\label{indices-of-abundance}}

Four standardized indices of abundance were generated for each of the
six species, one each for the data filtering method (drift-level
habitat-informed, drift-level Stephens-MacCall, trip-level
Stephens-MacCall) and an area-weighted index from the habitat-informed
drift-level data. All indices were modeled using Bayesian generalized
linear models (GLMs) and the delta GLM method
\citep{Lo:1992:IRA, Stefansson:1996:AGS}. The delta GLM method is
commonly used to standardize catch-per-unit effort for stock assessments
{[}citations{]}. The delta method models the the data with two separate
GLMs; one for the probability of encountering the species of interest
from a binomial likelihood and a logit link function and the second
models the positive encounters with either gamma or lognormal error
structure. The error structure of the positive model was selected via
the Akaike Information Criterion (AIC) from models with the full suite
of considered explanatory variables.

The response variable for the positive models was angler-retained catch
per unit effort. For the indices modeled at the level of a drift, effort
was calculated as the number of angler hours fished on a drift. The
trip-level effort was calculated as angler days, using the average
number of observed anglers across all drifts on a trip.

To keep comparisons across data filtering methods similar, depth was not
considered as an explanatory variable in the habitat-informed index.
Depth is often a significant explanatory variable for rockfish species,
with many rockfish species and populations separated by depth
\citep{Love:2002:RNP}. Year was always included in as an explanatory
variable in model selection, even if it was not significant, because the
goal of the index of abundance was to extract the year effect. Other
explanatory variables considered for the habitat-informed index were
aggregated regions rocky reefs (categorical variable\emph{)} and wave (a
3-month aggregated period of time, e.g., January-March). The
area-weighted index also included a year/rocky reef interaction term,
even if it was not statistically significant, to allow us to weight the
index by the area of rocky reef. The regions of rocky reef were
aggregated differently for each species to ensure adequate sample sizes
to explore the year/rocky reef interaction.

Explanatory variables for the two indices using the data filtered using
Stephens-MacCall method (blind to habitat information at the drift- and
trip-level) included only year, wave and aggregated counties of landing.
California has 14 coastal counties north of Point Conception, 11 of
which were represented in these data. We aggregated the northern
counties of Del Norte, Humboldt and Mendocino into one region, Sonoma
and Marin counties just north of San Francisco into another region and
Alameda and San Francisco counties into a third region. The remaining
counties of San Mateo, Santa Cruz, Monterey and San Luis Obispo remained
unaggregated.

Model selection for the binomial and positive observation models was
based on AIC using the lme4 package in R, and unless very different
predictors were selected, the same predictors were used in each of the
two Bayesian models. The Bayesian models were run with 5,000 iterations
and weakly informative priors. Posterior predictive model checks were
examined for both the binomial and positive observation models,
including the predicted percent positive compared to the maximum
likelihood estimates. We constructed the final year index by multiplying
the back-transformed posterior draws from the binomial model with the
exponentiation of positive model draws, and taking the mean and standard
deviation for each year.

The area-weighted habitat-informed index was developed by extracting the
posterior draws of from each year and area combination of the binomial
and positive posterior predictions, and then summing across the product
of the back-transformed posteriors weighted by the fraction of total
area within each reef. To compare the indices across the three data
filtering methods and the area-weighted index, each index was scaled to
its mean value.

\hypertarget{results}{%
\section{Results}\label{results}}

\hypertarget{survey-data-and-habitat-based-filtering-1}{%
\subsection{Survey Data and Habitat-based
Filtering}\label{survey-data-and-habitat-based-filtering-1}}

The data sets were filtered for errors within the relational database
before analyses and the data used here reflects changes from the QA/QC
process that may not be reflected in data available directly from the
CDFW. Approximately 21\% of all the CDFW observed CPFV trips from
2004-2016 occurred north of Point Conception and it is important to note
that north of Bodega Bay, California the majority of charter boats are
smaller 6-pack vessel that may not have the capacity to carry a sampler
onboard. The addition of the Cal Poly onboard observer survey to the
CDFW survey increased the sample sizes of observed trips in San Luis
Obispo county by an average of 155\% from 2004-2016.

From 2004-2016 the drift-level data contained a total of 19,425 fishing
drifts, and after removing drifts with missing effort information (time
fished and/or observed anglers), 19,180 drifts remained. The filter for
fishing drifts and observed anglers resulted in fishing drifts lasting
between three and 96 minutes and three to 15 observed anglers, and
reduced the data to 18,591 fishing drifts. The remaining data filter for
depth resulted in a cutoff of 46.6 fathoms, and retained 18,405 drifts
based on average drift depth.

We defined 108 areas of rocky habitat at the finest scale within
California state waters from Point Conception to the California/Oregon
border. The 2 m resolution of the substrate shows the patchiness and
heterogeneity of the rocky substrate (insets A and B of
{[}\#fig:map{]}). While the location-specific data from the fishing
fleet is governed by confidentiality, a high proportion of the fishing
drifts were associated with rocky habitat. This was verified by the
distributions of the distance from rocky habitat for each of the six
species. The distance from rocky habitat cutoff for blue, China and
gopher rockfish was six meters, eight meters for vermilion rockfish, 14
meters for black rockfish and 16 meters for brown rockfish. This final
data filter resulted in xxx drifts for blue, China and gopher rockfish,
zxxxx for vermilion rockfish, xxx for black rockfish and xx for brown
rockfish.

After exploratory analyses and considering the the availability of data,
the areas rocky habitat were grouped into five regions to ensure
adequate sample sizes for developing indices of abundance
(Figure~\ref{fig-map}). While covering a small area (5\% of the rocky
habitat), the number of observed fishing drifts within state waters
around the Farallon Islands off the coast of San Francisco was high
enough to warrant keeping it as a separate area of rocky habitat. The
region defined from the California/Oregon to San Francisco encompasses
49\% of the total rocky habitat in state waters by area, but only 12\%
of the observed drifts fished in this area. Each of the four remaining
regions of rocky habitat defined from San Francisco to Point Conception
contained an average of 12\% of the available habitat. The CDFW
estimated fishing effort by district, which does not exactly align with
our areas of grouped reef habitat. Only considering the fishing effort
north of Point Conception, CDFW estimated an average of 9\% of the CPFV
from the California/Oregon border through Mendocino County, 38\% from
Sonoma through San Mateo County, and 53\% from Santa Cruz to Point
Conception.

The differences in latitudinal distribution of the six species is
apparent from the maps of percent of positive observations
(Figure~\ref{fig-percentpos}). Black rockfish are distributed north fo
San Francisco, a more northerly distribution reflected in the
aggregation of data from Santa Cruz and south, whereas brown rockfish is
distributed across coastal California. Percent postive catch generally
showed higher catches south of San Francisco for vermillion, gopher,
brown, and blue rockfish. Black rockish showed higher postive catches in
the north, while the percent of drifts retaining China rockfish were all
around low coastwide. The average CPUE was highest for blue rockish
between San Francisco south to Big Sur (Figure~\ref{fig-cpue}). Black
rockfish average CPUE higher in the north, while gopher rockfish CPUE
was generally consistent throught the coast, albiet slightly higher
south of Big Sur. China rockfish CPUE catch was typically low coastwide,
with slightly higher catch rates in the Farallon Island reefs.

The final aggregation of the reefs and total area within each region are
found in Table () and reflect the disribution and patterns in the visual
representation of commonness in the data. The fraction of drifts
retained for the indices of abundance was high for all six species (80\%
or greater), indicating that many of drifts within these data occurred
near areas of rocky habitat.

\hypertarget{stephens-maccall-data-filtering-1}{%
\subsection{Stephens-MacCall Data
Filtering}\label{stephens-maccall-data-filtering-1}}

A total of 19,425 drifts that aggregated to 2,252 trips were used for
the trip-level Stephens-MacCall filtering. In general, the co-occurring
species used for the Stephens-MacCall method were similar for the
drift-level and the trip-level data. We present the coefficients and
95\% confidence intervals for the species coefficients for black
rockfish and brown rockfish in Figure (fdfsfdj). The plots for the
remaining four species are available in the supplemental materials. The
confidence intervals were larger for the trip-level data and the
co-occuring species at the drift-level provide a refined look at species
that have positive coefficients. For black rockfish, a noticible
difference is the intercept. At the trip-level the intercept
(probability the only black rockfish is encountered) is uninformative
and at the drift-level the intercept is strongly negative. A higher
fraction of the co-occuring species provide uniformative information
(the 95\% confidence interval crosses zero) for the trip-level data than
the drift-level.

The percent of the samples retained for each data method differed by
species, but followed the general trend that the lowest percent of
samples were retained from the Stephens-MacCall filtering at the drift
level, ranging from 12\% of samples retained for China rockfish and 54\%
for blue rockfish. A much higher percent of samples were retained both
from the other two methods, with an average of 83\% of drifts retained
when habitat was included as a filter. Data filtering for the indices
with data aggregated to the trip-level and using the status quo of
retaining all positive observations resulted in a high proportion of
positive samples (0.70 - 0.86) for all species.

\hypertarget{indices-of-abundance-1}{%
\subsection{Indices of Abundance}\label{indices-of-abundance-1}}

All but three of the 24 indices of relative abundance were modeled with
a lognormal distribution. The trip-level indices for black, blue and
gopher rockfish were modeled using a gamma distribution. In general, the
larger increases and decreases in the indices were similar among the
four indices developed for each species (Figure
Figure~\ref{fig-black-indices}). The generalized approach used in this
paper to create indices with comparable methods resulted in different
results for each species. The area-weighted indices are reflective of
the total available habitat and use all of the available high resolution
habitat and fishing drift data. The effects of this can be seen in the
plots where the area-weighted indices depart from the habitat-informed
drift-level indices. For example, the effect of the area-weighting is
apparent for black rockfish in 2005, 2007, 2009 and 2013. For China
rockfish the habitat-informed indices present a more variable index,
whereas both the Stephens-MacCall filtered datasets are more similar.
For vermilion rockfish, while the trends are similar among all four
indices, the effect of area-weighting dampens the increase modelled from
the habitat-informed drift level data from 2004-2006.

China rockfish is the only species for which the trip-level index had
the lowest average coefficient of variation, which increased with the
the habitat-informed filtering (Table \citet{tab-CV}). For all other
species, the habitat-informed filtering resulted indices with a lower
average CV than the trip-level filtering. This is most apparent for
brown and gopher rockfish where the estimated error shrinks drastically
for all of the drift level indices versus the trip-level index (Figure
xx) T

The average CVs between the drift-level area-weighted index and the
drift-level habitat informed indices were similar, as expected, since
they both used the same data with the only difference being the
year:area interaction in the models. However, the average CV between
drift-level habitat-informed filtering and Stephens-MacCall filtering
for the drift-level data differed by species. The average CV for brown
rockfish from the Stephens-MacCall filtering was large (0.679) compared
to the habitat informed filtering (0.142).

\FloatBarrier

\hypertarget{discussion}{%
\section{Discussion}\label{discussion}}

Data were limited to the California coast north of Point Conception
(\(34^\circ 27^\prime N\)). The composition of the fish communities in
southern California differ, and the recreational fisheries are
fundamentally different, with a higher percentage of trips targeting
mixed species and pelagic and highly migratory species, as well as more
limited access to rocky habitat nearshore. Point Conception is a
biogeographic break (citation) and a number of stock assessments In
addition, complete habitat data are not available for areas in southern
California. The data were also temporally restricted to the years
2001-2016. Earlier and more recent data were excluded to preserve a data
set with the most consistent gear and depth regulations.

Habitat layers The characteristics and classification of the rocky
habitat into more specific substrate types, e.g., boulder vs pinnacle,
is available for a small fraction of the mapped area. Therefore, all
areas of rocky substrate are currently created equal. A number of video
surveys have shown habitat associations differ by species and ,

Oftentimes a captain will position the vessel adjacent to rocky habitat
so that the current allows the vessel to drift over the rocky habitat.

The Stephens-MacCall model was developed to approximate habitat for
recreational fisheries data with unknown fishing locations. The onboard
observer surveys coupled with the high resolution rocky reef habitat
maps remove the uncertainty in both fishing locations and the availalbe
habitat. While the Stephens-MacCall filter is useful in identifying
co-occurring or non-occurring species it assumes all effort was exerted
in pursuit of a single target species. The targeting of more than one
species or species complex (``mixed trips'') can result in co-occurrence
of species in the catch that do not truly co-occur in terms of habitat
associations informative for an index of abundance. This was clearly
shown in the differences between the trip-level Stephens-MacCall
filtering that relies on the information gathered from an entire trips
to the drift-level Stephens-MacCall filtering that reflects the species
encountered at a single location.

Both blue and black rockfish have high affinity to rocky habitat, but
occur higher off the bottom and are both schooling species. It is not
uncommon to have a a drift dominated by blue rockfish in central
California, or black rockfish further north. However, the
Stephen-MacCall approach does not account for this by modeling
presence/absence.

The choice of a threshold value to use as a data filter from the
Stephen-MacCall method should be reviewed to determine how sensitive an
index of abundance is to that method. The

people have been addressing SM questions and how to deal with space in
stock assessmetn for awhile

The majority of groundfish species targeted by the CPFV fleet north of
Point Conception during the time period of this study all have high
associations to rocky habitat. In this case, the Stephens-MacCall method
can be considered a proxy for habitat when the species of interest has
known associations. This can be expanded in areas where trips are known
to target species of interest, but no habitat data are available the
proportion of trips encountering the target species could be used as a
proxy for habitat. This does not hold for areas where multiple species
complexes are targeted on same trip, e.g, a multi-day trip may target
large pelagic species and once trip limits are reached, the trip may
focus on a secondary target, which is the case for the California CPFV
fleet fishing south of Point Conception.

The suite of six species that we modelled in this paper is a concrete
example of why habitat is important and also varies among the species.
The high proportion of retained drifts across species when using habitat
as a data filter indicates that hate majority of drifts occurred over,
or very close to, rocky habitat.

There are a number of key assumptions made when using the onboard
observer data in a stock assessment. A key assumption of the onboard
observer surveys is that fishing behavior remains the same when samplers
are not onboard the vessel. If a captain only fishes particular
locations or targets a different suite of species when a sampler is
onboard the vessel, additional bias is introduced in the data.\\
spatio-temporal modelling.

Versions of the indices filtered based on habitat were approved by the
Pacific Fisheries Management Council's Science and Statistical Committee
for use in the 2013 stock assessments and have been used all of the
stock assessment process since. Comparisons should not be drawn between
the indices presented here and the stock assessment documents as the
indices in this paper were simplified to develop direct comparisons
among methods. When filtering and modelling data for a stock assessment,
additional filtering steps would be taken, such as excluding areas where
species are rare, e.e., south of Santa Cruz for black rockfish. However,
this is also a function of the lower sampling rates along the coast
north of San Francisco.

Addtional factors not considered in the simplified models presented here
include the fact that the catch from the recreational CPFV fishery is
dependent on a number of factors including weather, distance from port,
the clientele preferences, angler experience and captain's knowledge.
These models also do not account for distance to the nearest port, which
has been shown to significantly impact the access to fish as well as
historical fishing pressure\ldots.In addition, in 2004 the CDFW
implemented spatial and temporal closures to the recreational nearshore
groundfish fishery.

The fishery-dependent indices of abundance undergo higher levels of
scrutiny during stock assessment reviews due to the nature of the data
being driven by fisher behavior. The one fishery-independent survey for
nearshore groundfish in California north of California tends to have
similar trends to the fishery-dependent indices for the shallower
nearshore species like gopher and China rockfish.

\emph{The influence of an index of abundance is sometime the can have a
large influence on end year estimation of stock status (find examples).}

accepted for management (China, gopher/black-and-yellow,
vermilion/sunset, blue/deacon, black, lingcod - cite assessments).

Recent studies have identified the need to investigate the assumptions
and uncertainty in relative indices of abundance from visual surveys
\citep{Bacheler:2015:ERA, Campbell:2015:CRA} and simulation studies
\citep{Siegfried:2016:ISA}.

Prioritize data for stock assessments \citep{Magnusson:2007:WMF}.

Stock synthesis weighting of indices based on CVs - is the CV tighter
for the fishery-independent survey to give it have an edge over the
onboard observer survey?

Composition data from recreational surveys had the largest impact on
simulation results, but individual survey components did not have
individual effects on benchmarks \citep{Siegfried:2016:ISA}.

\FloatBarrier

\hypertarget{tables}{%
\section{Tables}\label{tables}}

\begin{table}

\caption{Area of rocky habitat in state waters 
                    aggregated to levels modelled for each species. 
                  The shaded blocks for each species indicate which areas 
                  ere aggregated to ensure appropriate samples sizes to explore 
                  an area-weighted index.}
\centering
\begin{tabular}[t]{>{\raggedright\arraybackslash}p{1.8in}|>{\raggedleft\arraybackslash}p{.8in}|>{\raggedleft\arraybackslash}p{.8in}|>{\raggedleft\arraybackslash}p{.8in}|>{\raggedleft\arraybackslash}p{.8in}|>{\raggedleft\arraybackslash}p{.8in}}
\toprule
Rocky Reef Desginations & Blue rockfish \& Vermilion rockfish & Black rockfish & Brown rockfish & China rockfish & Gopher rockfish\\
\midrule
California border to San Francisco & 439.546 & 439.546 & 439.546 &  & \\
\cmidrule{1-4}
San Francisco to Santa Cruz & 108.424 & 108.424 &  & \multirow{-2}{.8in}{\raggedleft\arraybackslash 547.970} & \\
\cmidrule{1-3}
\cmidrule{5-5}
Farallon Islands & 50.252 &  & \multirow{-2}{.8in}{\raggedleft\arraybackslash 498.967} & 50.252 & \\
\cmidrule{1-2}
\cmidrule{4-5}
Moss Landing to Big Sur & 137.603 &  &  & 137.603 & \multirow{-4}{.8in}{\raggedleft\arraybackslash 735.825}\\
\cmidrule{1-2}
\cmidrule{5-6}
Big Sur to Morro Bay & 90.424 &  & \multirow{-2}{.8in}{\raggedleft\arraybackslash 228.027} &  & 90.424\\
\cmidrule{1-2}
\cmidrule{4-4}
\cmidrule{6-6}
Morro Bay to Point Conception & 112.264 & \multirow{-4}{.8in}{\raggedleft\arraybackslash 390.543} & 112.264 & \multirow{-2}{.8in}{\raggedleft\arraybackslash 202.688} & 112.264\\
\bottomrule
\end{tabular}
\end{table}

\begin{tabular}{>{\raggedright\arraybackslash}p{1.8in}rrr}
\toprule
Species & Drift-level habitat-informed & Drift-level SM-filtered & Trip-level\\
\midrule
Black rockfish & 0.886 & 0.252 & 0.408\\
Blue rockfish & 0.830 & 0.538 & 0.871\\
Brown rockfish & 0.855 & 0.243 & 0.490\\
China rockfish & 0.808 & 0.121 & 0.515\\
Gopher rockfish & 0.787 & 0.401 & 0.755\\
\addlinespace
Vermilion rockfish & 0.799 & 0.382 & 0.821\\
\bottomrule
\end{tabular}

\begin{table}

\caption{The average Coefficient of Variation (CV) for 
                  each index of abundance, where SM-filtered is the 
                  Stephens-MacCall filtering.}
\centering
\begin{tabular}[t]{lrrrr}
\toprule
Species & Drift level Area-weighted & Drift level habitat-informed & Drift level SM-filtered & Trip-level SM-filtered\\
\midrule
Black rockfish & 0.443 & 0.449 & 0.364 & 0.671\\
Blue rockfish & 0.134 & 0.142 & 0.099 & 0.257\\
Brown rockfish & 0.242 & 0.240 & 0.679 & 0.858\\
China rockfish & 0.320 & 0.301 & 0.233 & 0.151\\
Gopher rockfish & 0.179 & 0.183 & 0.138 & 0.626\\
\addlinespace
Vermilion rockfish & 0.152 & 0.178 & 0.133 & 0.238\\
\bottomrule
\end{tabular}
\end{table}

\FloatBarrier

\hypertarget{figures}{%
\section{Figures}\label{figures}}

\begin{figure}

{\centering \includegraphics{figures/map_2.jpg}

}

\caption{\label{fig-map2}A example of the high resolution bathymetric
data and components used to create the rocky reef habitat layer for
groundfish (far right panel).}

\end{figure}

\begin{figure}

{\centering \includegraphics{figures/map.jpg}

}

\caption{\label{fig-map}A maps of California state waters north of Point
Conception colored by the aggregated areas of rocky reef habitat,
including inset A depicting the rocky reef habitat in relation to 3 nm
state water boundary state waters and inset B showing the high
resolution rocky habitat in the area.}

\end{figure}

\begin{figure}

{\centering \includegraphics{figures/percentpositives_map.jpg}

}

\caption{\label{fig-percentpos}The percent of drifts that retained the
target species, within grouped areas of rocky habitat over all years of
the time series. The grey dashed lines represent the aggregated rocky
habitat used to develop an index of abundance.}

\end{figure}

\begin{figure}

{\centering \includegraphics{figures/CPUE_map.jpg}

}

\caption{\label{fig-cpue}The average CPUE across all years of the time
series for each of the six species. The grey dashed lines represent the
aggregated rocky habitat used to develop an index of abundance.}

\end{figure}

\begin{figure}

\begin{minipage}[t]{0.50\linewidth}

{\centering 

\raisebox{-\height}{

\includegraphics[width=2.7in,height=\textheight]{figures/black_trip_sm.png}

}

}

\subcaption{\label{fig-black-tripsm}Black rockfish trip-level}
\end{minipage}%
%
\begin{minipage}[t]{0.50\linewidth}

{\centering 

\raisebox{-\height}{

\includegraphics[width=2.7in,height=\textheight]{figures/black_drift_sm.png}

}

}

\subcaption{\label{fig-black-driftsm}Black rockfish drift-level}
\end{minipage}%
\newline
\begin{minipage}[t]{0.50\linewidth}

{\centering 

\raisebox{-\height}{

\includegraphics[width=2.7in,height=\textheight]{figures/brown_trip_sm.png}

}

}

\subcaption{\label{fig-brown-tripsm}Brown rockfish trip-level}
\end{minipage}%
%
\begin{minipage}[t]{0.50\linewidth}

{\centering 

\raisebox{-\height}{

\includegraphics[width=2.7in,height=\textheight]{figures/brown_drift_sm.png}

}

}

\subcaption{\label{fig-brown-driftsm}Brown rockfish drift-level}
\end{minipage}%

\caption{\label{fig-sm}Examples of the species coefficients and 95\%
confidence intervals for the Stephens-MacCall filtering for black
rockfish and brown rockfish for the trip-level and drift-level data.}

\end{figure}

\begin{figure}

\begin{minipage}[t]{0.50\linewidth}

{\centering 

\raisebox{-\height}{

\includegraphics[width=3in,height=\textheight]{figures/black_indices.png}

}

\caption{\label{fig-black-indices}Black rockfish}

}

\end{minipage}%
%
\begin{minipage}[t]{0.50\linewidth}

{\centering 

\raisebox{-\height}{

\includegraphics[width=3in,height=\textheight]{figures/blue_indices.png}

}

\caption{\label{fig-blue-indices}Blue rockfish}

}

\end{minipage}%
\newline
\begin{minipage}[t]{0.50\linewidth}

{\centering 

\raisebox{-\height}{

\includegraphics[width=3in,height=\textheight]{figures/brown_indices.png}

}

\caption{\label{fig-brown-indices}Brown rockfish}

}

\end{minipage}%
%
\begin{minipage}[t]{0.50\linewidth}

{\centering 

\raisebox{-\height}{

\includegraphics[width=3in,height=\textheight]{figures/china_indices.png}

}

\caption{\label{fig-china-indices}China rockfish}

}

\end{minipage}%
\newline
\begin{minipage}[t]{0.50\linewidth}

{\centering 

\raisebox{-\height}{

\includegraphics[width=3in,height=\textheight]{figures/gopher_indices.png}

}

\caption{\label{fig-gopher-indices}Gopher rockfish}

}

\end{minipage}%
%
\begin{minipage}[t]{0.50\linewidth}

{\centering 

\raisebox{-\height}{

\includegraphics[width=3in,height=\textheight]{figures/vermilion_indices.png}

}

\caption{\label{fig-vermilion-indices}Vermilion rockfish}

}

\end{minipage}%
\newline
\begin{minipage}[t]{0.50\linewidth}

{\centering 

Scaled indices of abundance for four data filtering methods explored.

}

\end{minipage}%

\end{figure}


  \bibliography{bibliography.bib}


\end{document}
