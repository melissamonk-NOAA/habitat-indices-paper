\documentclass[preprint, 3p,
authoryear]{elsarticle} %review=doublespace preprint=single 5p=2 column
%%% Begin My package additions %%%%%%%%%%%%%%%%%%%

\usepackage[hyphens]{url}

  \journal{Fisheries Research} % Sets Journal name

\usepackage{lineno} % add

\usepackage{graphicx}
%%%%%%%%%%%%%%%% end my additions to header

\usepackage[T1]{fontenc}
\usepackage{lmodern}
\usepackage{amssymb,amsmath}
\usepackage{ifxetex,ifluatex}
\usepackage{fixltx2e} % provides \textsubscript
% use upquote if available, for straight quotes in verbatim environments
\IfFileExists{upquote.sty}{\usepackage{upquote}}{}
\ifnum 0\ifxetex 1\fi\ifluatex 1\fi=0 % if pdftex
  \usepackage[utf8]{inputenc}
\else % if luatex or xelatex
  \usepackage{fontspec}
  \ifxetex
    \usepackage{xltxtra,xunicode}
  \fi
  \defaultfontfeatures{Mapping=tex-text,Scale=MatchLowercase}
  \newcommand{\euro}{€}
\fi
% use microtype if available
\IfFileExists{microtype.sty}{\usepackage{microtype}}{}
\usepackage[]{natbib}
\bibliographystyle{plainnat}

\ifxetex
  \usepackage[setpagesize=false, % page size defined by xetex
              unicode=false, % unicode breaks when used with xetex
              xetex]{hyperref}
\else
  \usepackage[unicode=true]{hyperref}
\fi
\hypersetup{breaklinks=true,
            bookmarks=true,
            pdfauthor={},
            pdftitle={Comparison of data filtering methods for indices of abundance from a recreational fishery survey},
            colorlinks=false,
            urlcolor=blue,
            linkcolor=magenta,
            pdfborder={0 0 0}}

\setcounter{secnumdepth}{5}
% Pandoc toggle for numbering sections (defaults to be off)


% tightlist command for lists without linebreak
\providecommand{\tightlist}{%
  \setlength{\itemsep}{0pt}\setlength{\parskip}{0pt}}






\begin{document}


\begin{frontmatter}

  \title{Comparison of data filtering methods for indices of abundance
from a recreational fishery survey}
    \author[Southwest Fisheries Science Center]{Melissa Hedges Monk}
   \ead{melissa.monk@noaa.gov} 
    \author[UCSC]{Rebecca R Miller}
   \ead{rebecca.miller@noaa.gov} 
    \author[UCSC]{Grant Waltz}
   \ead{fkdjsl@example.com} 
    \author[Cal Poly]{Dean Wendt}
   \ead{fkdjsl@example.com} 
      \affiliation[Southwest Fisheries Science Center]{Southwest
Fisheries Science Center, Fisheries Ecology Division, 110 McAllister
Way, Santa Cruz, CA 95060}
    \affiliation[UCSC]{Department, Street, City, State, Zip}
    \affiliation[Cal Poly]{Department, Street, City, State, Zip}
    \cortext[cor1]{Corresponding author}
  
  \begin{abstract}
  This is the abstract.

  It consists of two paragraphs.
  \end{abstract}
    \begin{keyword}
    surveys \sep stock assessment \sep 
    index of abundance
  \end{keyword}
  
 \end{frontmatter}

\hypertarget{introduction}{%
\section{Introduction}\label{introduction}}

Fisheries stock assessments rely on a wide range of data to model a
fishery's population dynamics. Catch data is a primary input to stock
assessments and informs the overall magnitude of the stock. Catch data
are often input with the assumption that the removals are known with
absolute precision, i.e., there is no error associated with removals. A
secondary data stream is an index of abundance that provides information
on the size of the population. An absolute index of abundance is a
census of a fish stock that is oftentimes input as a single year due to
the high cost associated with determining total fish abundance within an
area (include example).

More often, an index of abundance is a relative measure of the
population over time and requires a time series, of four to five years
at a minimum. Fisheries survey and catch data are used to develop
standardized indices of abundance that inform fisheries stock assessment
models \citep{Maunder2004}.Fishery-independent data collected from
standardized survey designs provide are preferred when creating an index
of abundance to represent the trend in a fisheries population.
Fishery-independent surveys\ldots.. However, fishery-independent surveys
are costly, labor intensive and often require a long time series to be
considered informative in fisheries stock assessments.

When fishery-independent data are not available, assessors try to make
best used of the best available data, which may often only include
fishery-dependent data.

There are both advantages and disadvantages that must be considered when
using to fishery-dependent data. Fishery-dependent data are collected
directly from the the fishery and are less costly than the whose
operations are not constrained by sampling designs, but dependent on the
behaviors of the captain and vessel and, in the case of recreational
trips, customer preference.

Fishery-dependent data are only collected from areas legally open areas
can be collected, i.e., areas closed to fishing are not sampled. In
California, this includes a network of marine protected areas (MPAs),
rockfish conservation areas (RCAs) developed based on depth closures,
and varying seasonal and depth closures that vary temporally and
spatially along California's coastline. Fishery-independent surveys are
conducted using a scientific study design and, depending on the study,
are not always confined to the same regulations as the commercial and
recreational fishing sectors. In an ideal situation, both
fishery-dependent and fishery-independent surveys would used to inform
the stock assessment model.

Catch per unit effort (CPUE) is a common metric collected from
fishery-dependent or fishery-independent surveys, with the latter
providing unbiased data. Depending on the stock assessment model and the
available data for a particular fish stock, an index of stock status can
have a large influence on end year estimation of stock status (find
examples).

An index of relative abundance assumes that changes in the index are
proportional to changes of abundance in the population
\citep{Harley2001}.

Fishery-dependent surveys sample the fishing fleets and are subject to
potential sampling biases. The sampling is dependent on the fishing
boat's behavior, which is to maximize catch. Sampling of the fishing
fleet is often opportunistic based on the availability of samplers and
the availability of trips to sample. Sampling the fisheries can also be
constrained to the current regulations, which may prohibit the retention
of a species or fishing at certain depths, i.e., California Department
of Fish and Wildlife has varying spatial and temporal depth and season
closures implemented through six management regions. There is also a
fairly new network of Marine Protected Areas (MPAs) designated from
2007-2012 that prohibit recreational fishing, and are therefore areas no
longer sampled by the recreational fishing fleet. However, the advantage
to fishery-dependent sampling the reduced program cost compared to a
more intensive fishery-independent survey.

This data series evaluated in this analysis is part of the onboard
observer program, which collects location- and species-specific CPUE
information for the recreational fishing fleet (Monk et al.~2014)

The Pacific Fishery Management Council manages groundfish off the West
Coast of the United States under the Groundfish Fishery Management Plan
(FMP). The FMP includes 64 species of rockfish, \_\_\_ of which do not
have full stock assessments. Many of these species, especially those
nearshore, are assessed using multiple assessment models to represent
areas with distinct fishing histories, communities regulations. Along
the U.S. West Coast, even if the stock assessment is categorized as data
rich, oftentimes the only index of abundance available is from a
fishery-dependent CPUE time series of observed recreational angler catch
rates \citep{Cope2013}.

This paper focuses on the development of methods to select samples from
a larger survey that represent the appropriate effort directed at a
species of interest. We developed standardized indices of abundance
based on different data filtering methods for a fishery-dependent survey
of the recreational fishing fleet.

\hypertarget{methods}{%
\section{Methods}\label{methods}}

\hypertarget{survey-data}{%
\subsection{Survey Data}\label{survey-data}}

The California Department of Fish and Wildlife (CDFW) has conducted a
fishery-dependent onboard observer survey of the Commercial Passenger
Fishing Vessel (CPFV or party/charter boat) fleet since 1999. Since
2004, the survey became part of the California Recreational Fisheries
Survey (CFRS). Sampling effort for groundfish-targeted CPFV trips was
distributed in proportion to fishing effort. In California,
approximately xx\% of the recreational CPFV effort is north of Point
Conception. Vessels are required to carry observers if requested, but
observers may not be allowed on a vessel if the vessel is at full
capacity. This is more common in northern California where a number of
charter boats are smaller 6-pack vessels with limited capacity.

Groundfish-targeted CPFV trips were sampled opportunistically as CPFV
participation is voluntary and sampling effort was distributed in
proportion to fishing effort. In California, xx\% of the recreational
CPFV effort is north of Point Conception. Observers may not be allowed
on a vessel if the vessel is at full capacity, which is more common in
northern California where a number of charter boats are smaller 6-pack
vessels with limited capacity.

On a trip, observers recorded information for each fishing drop, each
time lines were in the water. Just prior to the start of each fishing
drop, the sampler selected a subset of anglers to observe, at maximum of
15 anglers per fishing drop. The sampler recorded all fish encountered
(retained and discarded) by the subset of anglers as a group. Samplers
also recorded the time fished (starting when the captain announced
``Lines down'' to when the captain instructed anglers to reel lines up),
GPS coordinates of the fishing drop (start and/or end
latitude/longitude), and minimum and maximum bottom depth. Fish
encountered by the group of observed anglers were recorded to the
species level as either retained or discarded, providing a count of each
species at a particular location. Discarded fish were measured for
length and some portion of retained fish were measured as part of a
different CRFS Sampling program. The catch and fishing time of an
individual angler were not recorded. Additional details can be found in
Monk et al. \citeyearpar{Monk2014}.

In 2001, the California Polytechnic State University Institute of Marine
Science, San Luis Obispo (Cal Poly) began conducting a supplemental
onboard observer program of the CPFV fleet based in Port Avila and Port
San Luis along the Central Coast. Protocols for the Cal Poly survey were
the same as the CDFW survey, with the exception that Cal Poly measured
retained and discarded fish from observed anglers.

A common phenomenon of ecological data is the high proportion of zero
observations across samples and the question as to whether the sampling
occurred within the species' habitat and the species was not observed or
if the sampling occurred outside of the species' habitat (structural
zeroes).\\
Fisheries survey data are often subset to exclude structural zeroes
using the Stephens-MacCall method, which looks at the species
composition of co-occurring species. However, the onboard observer
survey collected location-specific information on each observer fish
encounter. To subset the onboard observer survey data and exclude
structural zeroes, we used the positive catch locations as a proxy for
suitable habitat.

\hypertarget{species}{%
\subsubsection{Species}\label{species}}

We explored the methods described in the following sections to develop
indices of abundance for fourteen species or species complexes of
management interest: black rockfish (\emph{Sebastes melanops}), blue and
deacon rockfish complex (\emph{Sebastes mystinus}, \emph{Sebastes
diaconus}), brown rockfish (\emph{Sebastes auriculatus}), China rockfish
(\emph{Sebastes nebulosus}), gopher rockfish (\emph{Sebastes carnatus}),
rosy rockfish (\emph{Sebastes rosaceus}), and vermilion and sunset
rockfish complex (\emph{Sebastes miniatus}/\emph{Sebastes crocotulus}).
Species complexes consist of two cryptic species that may or may not be
genetically distinct, but cannot be assessed separately for various
reasons including the inability to separate catch histories between
species or difficulty of visual species identification. Gopher rockfish
was assessed as part of a species complex with black-and-yellow rockfish
(\emph{Sebastes chrysomelus}) in 2019, but are visually identifiable
\citep{Monk2019}.

Versions of the area-weighted habitat index of relative abundance were
approved by the Pacific Fisheries Management Council's SSC for use in
stock assessments in 2013 have been used in xxx assessments accepted for
management (China, gopher/black-and-yellow, vermilion/sunset,
blue/deacon, black, lingcod - cite assessments).

\hypertarget{treatment-of-data}{%
\subsubsection{Treatment of Data}\label{treatment-of-data}}

The onboard observer data provide a high-resolution of catch, effort and
the ability to map the fishing drops to fine-scale habitat data. This
paper explores methodological differences in data treatment to see what
we gain by having the high-resolution habitat data and using that as a
mechanism to filter out trips that are not targeting the species of
interest. To do this, we first aggregated the drop-level species
encounter data to the trip-level to mimic the collection of dockside
data. Effort (angler hours) was calculated for each drop and summed
across a trip to estimate total effort for the trip. Trip-level data
were then filtered using the Stephens-MacCall approach and three
different data selection methods were applied using the Stephens-MacCall
results (see description below). In the resulting indices of abundance,
the only spatial covariate explored was the county of landing.

The second filtering approach used the high resolution drop-level data,
but assumed no available habitat data. The percent of groundfish
encountered during a drop was assumed as a proxy for habitat. The third
approached used the fishing drop-level data, incorporated habitat as a
filter for data selection, and applied area weights to the index, using
the total area of rocky habitat as the weights. In addition, all of
these approaches were applied with and with out the supplemental data
from the Cal Poly observer program to illustrate the effect of
additional data on indices for species with population ranges centered
in central California.

The onboard observer data provide a suite of data including catch,
effort, bottom depth, and the latitude/longitude of each fishing drop.
This paper explores methodological differences in data treatment to
determine changes in trends in indices and the associated error among
three alternative assumptions and data filtering strategies. All of the
methods described below started with the same subset of drifts from the
onboard observer data, restricted to state waters and the years
2001-2016. In the case of application to stock assessments, all
potential data are explored, which may be why trends in indices differ
in this paper than what has previously been published in stock
assessments. Since the most recent stock assessments in 2021, the data
have undergone a major quality assurance effort by the authors.

The majority of available recreational fishing data are collected by
sampling anglers and vessels dockside, after the completion of a fishing
trip.We mimicked the collection of dockside data by aggregating all of
the fish encountered within a single trip and summing the effort among
drifts. In this case, each trip is a sampling unit. Trip level data were
then filtered using the Stephens-MacCall approach (see description
below), and only county was used as a spatial covariate in the indices.
This approach was applied with and without the supplemental data
collections by Cal Poly.

The second approach used the high resolution drift data, but assumed no
available habitat data and applied the Stephens-MacCall filtering
approach, again with and without the supplemental sampling data from Cal
Poly.

The third approached used the fishing drop level data, incorporated
habitat as a filter for data selection, and compares indices with and
without area-weighting using the area of hard substrate within a region
as a proxy for habitat.

All indices of abundance were coded in R and the Bayesian analyses were
conducted using the rstanarm package.

Analyses were limited to the California coast north of Point Conception
(\(34^\circ 27^\prime N\)). The composition of the fish communities in
southern California differ, and the recreational fisheries are
fundamentally different, with a higher percentage of trips targeting
mixed species and pelagic and highly migratory species, as well as more
limited access to rocky habitat nearshore. Point Conception is a
biogeographic break (citation) and a number of stock assessments In
addition, complete habitat data are not available for areas in southern
California. The data were also temporally restricted to the years
2001-2016. Earlier and more recent data were excluded to preserve a
dataset with the most consistent gear and depth regulations.

\hypertarget{stephens-maccall-data-selection-filtering}{%
\paragraph{Stephens-MacCall Data Selection
Filtering}\label{stephens-maccall-data-selection-filtering}}

The trip-level index uses all available data before any filtering was
done to exclude individual drifts with missing effort or location data.
The trip-level effort was calculated as angler days, using the average
number of observed anglers across all drifts on a trip. This imitates
the method for which effort is calculated for the observed catch from
CRFS angler interviews.

The Stephens-MacCall \citeyearpar{Stephens2004} filtering approach was
used to predict the probability of encountering the target species,
based on the species composition of the catch in a given trip. The
method uses presence/absence data within a logistic regression to
identify the probability of encountering a target species given the
presence or absence of other predictor species. This method is commonly
used to filter data that are collected dockside after a vessel returns
to port. Prior to applying the Stephens-MacCall filter, we identified
potentially informative predictor species, i.e., species with sufficient
sample sizes and temporal coverage (present in at least 5\% of all
trips) to inform the binomial model. The remaining species all
co-occurred with the target species in at least one trip and were
retained for the Stephens-MacCall logistic regression. Coefficients from
the Stephens-MacCall analysis (a binomial GLM) are positive for species
that are more likely to co-occur with the target species, and negative
for species that are less likely to be caught with target species.

While the filter is useful in identifying co-occurring or non-occurring
species assuming all effort was exerted in pursuit of a single target,
the targeting of more than one species or species complex (``mixed
trips'') can result in co-occurrence of species in the catch that do not
truly co-occur in terms of habitat associations informative for an index
of abundance. Stephens and MacCall \citeyearpar{Stephens2004}
recommended including all trips above a threshold where the false
negatives and false positives are equally balanced. However, this does
not have any biological relevance and for this particular data set where
trained observers identify all fish. We assume that if the target
species was encountered, the vessel fished in appropriate habitat.

Stephens and MacCall \citeyearpar{Stephens2004} proposed filtering
(excluding) trips from the index standardization based on a criterion of
balancing the number of false positives and false negatives. False
positives (FP) are trips that are predicted to encounter the target
species based on the species composition of the catch, but did not.
False negatives (FN) are trips that were not predicted to encounter the
target species, given the catch composition, but caught at least one.
The trips selected using this criteria were compared to an alternative
method including all the ``false positive'' trips, regardless of the
probability of encountering the target species. The catch included in
this index and in the angler interviews collected by CDFW is
sampler-examined and the samplers are well trained in species
identification. Therefore, we make the assumption that species were
positively and correctly identified. Three data selection methods were
applied to the Stephens-MacCall method, the data selection method
proposed in the original manuscript to balance the false negatives and
the false positives, retention of all positive encounters and exclude of
only false negatives, and the method described in (xxxx).

Started with 2,252 trips that retained at least one fish.

A Stephens-MacCall filter was also applied to the data with each
individual drift as a sample. The

used a cutoff of 1\% of all drifts for the drift-level Stephens-MacCall
analysis, minimum drifts of 186

\hypertarget{data-selection-including-habitat}{%
\paragraph{Data Selection Including
Habitat}\label{data-selection-including-habitat}}

Drift level data selection was Total drifts is 19,425 and after removing
drifts with missing effort information left with 19,180

Upper and lower 1\% of the data removed for fish time and observed
anglers, leaving drift times between 3 and 96 minutes and observed
anglers between three and 15 anglers, and \textless= 4 leaving 18,591
drifts.

For indices incorporating habitat information, we filtered the depths to
ensure that appropriate information was used. We did not use depth
within the indices as a predictor. The fishery was closed deeper than 40
fathoms for the entire time period, and the additional 6 fathoms is
within the scope of error depending on the bottom habitat.To remove
drifts that may not have targeted groundfish, we removed drifts deeper
than the 99\% quantile, 46.6 fathoms, retaining 18,405 drifts.

The distance from rocky habitat composed the last filter for the habitat
and area-weighted indices. We retained 90\% of the drifts with the
target species from the cumulative frequency table of distance to rocky
habitat. The cutoff for blue, China and gopher rockfish was six meters,
eight meters for vermilion rockfish, 14 meters for black rockfish and 16
meters for brown rockfish.

ReefDistance cutoffs used Black 14 m Blue 6 m Brown 16 m China 6 m
Gopher 6 m Vermilion 8 m

\hypertarget{indices-of-abundance}{%
\paragraph{Indices of Abundance}\label{indices-of-abundance}}

Standardized indices of abundance were generated for each data filtering
method.

All indices were modeled using a Bayesian genearlized linear models
(GLMs). The onboard observer data were analyzed using the delta method
with two generalized linear models (delta-GLM). The first GLM models the
probability of encountering the species of interest with a binomial
likelihood and a logit link function. The second GLM models the positive
encounters with either gamma or lognormal errors structure.

We explored the possibility of area-weighted indices, using the area of
the reefs as the weighting scheme.

Indices of abundance modeled the catch per unit effort (angler hours)
and possible covariates trip-level data were 3-month wave and county of
landing. Covariates considerd for the drop-level data included,
aggregated reef area, 3-month wave, dep and

Standardized indices of abundance were generated for each data filtering
method using generalized linear models and methods approved for use in
West Coast groundfish stock assessments. All indices were modeled using
Bayesian generalized linear models (GLMs); species with high positive
encounter rates were modeled with a negative biniomial and species with
lower encounter rates were modeled using a delta method \citep{Lo1995}.
The delta method models the data with two separate GLMs; one for the
probability of encountering the species of interest from a binomial
likelihood and a logit link function and the second GLM models the
positive encounters with either gamma or lognormal error structure. The
gamma or lognomal model was chosen by AIC from the full model.

Indices of abundance modeled the catch per unit effort (angler hours)
and possible covariates.

Year was always included in the mo trip-level data were 3-month wave and
county of landing. Covariates considerd for the drop-level data
included, aggregated reef area, 3-month wave, depth, and xxxx. We
explored the possibility of area-weighted indices, using the area of the
reefs as the weighting scheme.

All indices of abundance were coded in R and the Bayesian analyses were
conducted using the rstanarm package.

\hypertarget{habitat-data}{%
\paragraph{Habitat Data}\label{habitat-data}}

We identified rocky habitat and defined reefs as potential habitat for
rockfish in California from multiple bathymetic data sources. Bathymetry
within California state waters north of Point Conception
(\(34^\circ 27^\prime N\)) was mapped at a resolution of 2 m by the
California Seafloor Mapping Program (CSMP).Rough and smooth substrate
was identified by CSMP using 2 rugosity indices based upon bathymetric
data, surface:planar area, and vector ruggedness measure (VRM). We
considered areas identified as `rough' as reef habitat. While there were
fishing drops outside of state waters, we limited data for the
comparisons presented in this paper to state waters with known habitat.

Individual reefs at the finest scale were defined as raster cells of
rough habitat greater than 200 m apart. The distance was chosen based on
evidence that a number of nearshore rockfish exhibit site fidelity and a
number of tagging studies have recaptured close to original capture
sites \citep{Lea1999, Matthew1990, Hannah2011, Hannah2012}. If raster
cells representing hard substrate were contiguous (not separated by soft
habitat by greater than 200 m) it remained intact, no matter how large
the reef. Reefs were further defined with a 5 m buffer to account for
potential error in positional accuracy.

Individual drifts were assigned to reefs based on the recorded start
location, given that the end locations were not always available. Reefs
within predetermined larger regions were designated to gain appropriate
sample sizes needed for modelling and the areas of the hard habitat were
summed.

\hypertarget{results}{%
\section{Results}\label{results}}

\hypertarget{discussion}{%
\section{Discussion}\label{discussion}}

Recent studies have identified the need to investigate the assumptions
and uncertainty in relative indices of abundance from visual surveys
(Bacheler and Shertzer 2015, Campbell et al.~2015, Schobernd et
al.~2013) and simulation studies (Siegfried et al.~2016).

Magnusson and Hilborn 2007 - prioritize data for stock assessments

Stock synthesis weighting of indices based on CVs - is the CV tighter
for the fishery-independent survey to give it have an edge over the
onboard observer survey?

CDFW sampler manual - ``10 anglers should be the target number of
observed anglers''

encompass the entire range of the species. However, the point of the
exercise is to compare the two methods and these surveys are sampling
the same habitats in the SCB

Survey indices can be either absolute or relative. In the case of an
absolute index of abundance, the entire population within the sampling
area is accounted for and the index also provides information on the
density of the fish species within that area as well as aid in scaling
the population size within the stock assessment model. Most indices of
abundance are relative due to the fact that the entire population within
the survey area was not observed. Estimates of absolute abundance are
difficult to obtain, especially for cryptic rockfishes. The cowcod
(\emph{Sebastes levis}) stock assessments is one of the only West Coast
stock assessments that has incorporated an estimate of absolute
abundance, derived from a visual survey {[}\citet{Piner} et al.~2005).
The majority of stock assessments include one or more index of relative
abundance.

Composition data from recreational surveys had the largest impact on
simulation results, but individual survey components did not have
individual effects on benchmarks (Siegfried et al.~2016).\\
The onboard observer surveys decrease the amount of uncertainty, but
relative to a fishery-independent survey, is still high\ldots.

A key assumption of the onboard observer programs is that fishing
behavior remains the same when observers are not onboard the vessel. If
a captain only fishes particulat locations or targets a different suite
of specides when an observer is onboard the vessel, additional bias is
introduced in the data

\hypertarget{tables}{%
\section{Tables}\label{tables}}

\hypertarget{figures}{%
\section{Figures}\label{figures}}

\renewcommand\refname{References}
\bibliography{mybibfile.bib}


\end{document}
